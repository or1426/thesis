
%UoY:

%The abstract shall follow the title page. It shall provide a synopsis of the thesis, stating the nature and scope of work undertaken and the contribution made to knowledge in the subject treated. It shall appear on its own on a single page and shall not exceed 300 words in length. The abstract of the thesis may, after the award of the degree, be published by the University in any manner approved by the Senate, and for this purpose, the copyright of the abstract shall be deemed to be vested in the University.


\chapter*{Abstract}
\addcontentsline{toc}{chapter}{Abstract}
%\begin{SingleSpace}
This thesis addresses two forms of quantum uncertainty. The first, preparation uncertainty, is an expression of the fact that there are sets of observables for which the induced probability distributions are not simultaneously sharp in any state. We exactly characterise the preparation uncertainty regions for several finite dimensional case studies, including a new derivation of the preparation uncertainty region for the Pauli observables of qubits, and two qutrit case studies which have not previously been addressed in the literature.\\
We also consider the preparation uncertainty for observables relevant to multi-slit interferometry. We characterise the appropriate observables by their covariance properties. For one covariance condition the observables turn out to be equivalent to the position and momentum observables for a particle confined to move on a one dimensional ring, whilst for another they are equivalent to the canonical observables of a particle confined to move in a one dimensional ``box'', with infinite potential walls. A full characterisation of the uncertainty region for the particle on a ring is known. For the box we give bounds on the boundary of the region, and show that our upper bound is exact in an interval.\\
In the second part we turn our attention to measurement uncertainty, exploring the space of compatible joint approximations to incompatible target observables. We prove a quite general theorem, showing that for a broad class of figures of merit, the optimal compatible approximations to covariant targets are themselves covariant. This substantially simplifies the problem of determining measurement uncertainty regions for covariant observables, since the space of covariant compatible approximations is smaller than the space of all compatible approximations.\\
We employ this theorem to derive measurement uncertainty regions for three mutually orthogonal Pauli observables, and for the quantum Fourier pair acting in any finite dimension.
%\end{SingleSpace}
\clearpage