\let\textcircled=\pgftextcircled
\chapter{Preliminaries}
\label{chap:prelim}


\bigskip

\section{Mathematical background of quantum theory}

We will be considering separable Hilbert spaces over the complex numbers, where separability is taken to mean the existence of a countable orthonormal basis. We will adopt the convention that the inner product, denoted $\langle\cdot|\cdot\rangle$, is linear in the second slot and conjugate-linear in the first. Such an inner product naturally induces a norm $\norm{\boex} \defeq \sqrt{\langle \boex| \boex\rangle}$. We will use the notation $\bops[V][W]$ the space of continuous linear maps between normed vector spaces $V$ and $W$, over the same field, omitting the second argument if the two spaces are the same. This space is once more a vector space over the same field, with addition and scalar multiplication being defined pointwise and inherits a natural norm 
\begin{align}
  \norm{\cdot}&:\bops[V][W]\to \K\\
  \norm{T} &\defeq \sup_{\substack{\bov\in V \\ \norm{\bov}\leq 1}}\norm{T\bov}, \label{eqn:operator-norm}
\end{align}
under which it is a Banach space \coma{cite R\&S?}. A particularly important special case is the space of maps $\bops[V][\K]$, where $\K$ is the field underlying $V$, which we will denote $V^*$, and is commonly called the topological dual of $V^*$. A celebrated theorem due to Riesz \cite{riesz-representation-riesz} and Fr{\'e}chet \cite{riesz-representation-frechet}, commonly called the Riesz representation theorem, states that Hilbert space over the reals is isomorphic to its topological dual, and one over the complexes is anti-isomorphic to its dual. More explicitly every continuous linear functional on a Hilbert space is of the form $\Psi_{\bow} :\bov\mapsto \langle{\bow} |\bov\rangle$, for some fixed $\bow\in V$, and the map $\Psi_{\bow}$, defined in this way is a continuous linear functional for every $\bow\in V$.

Given two vector spaces, $V$, $W$ over the same field $\K$ we define the \emph{tensor product} $V\times W$ by first defining the free vector space over $\K$, $F(V\otimes W)$ generated by the tuples $(\bov, \bow)$ for each $\bov\in V$ and $\bow\in W$. To obtain the tensor product we then quotient by the equivalence relation $\sim$ such that
\begin{align}
  (\bov,\bow) + (\bov^\prime, \bow) &\sim (\bov+ \bov^\prime, \bow)\\
  (\bov,\bow) + (\bov, \bow^\prime) &\sim (\bov, \bow+\bow^\prime)\\
  c(\bov,\bow) &\sim (c\bov,\bow)\sim (\bov,c\bow),
\end{align}
for all $\bov, \bov^\prime\in V$, $\bow,\bow^\prime \in W$, $c\in \K$. We denote the equivalence class $[(\bov,\bow)]$ by $\bov\otimes \bow$.

We define the \emph{tensor product} $V\otimes W$ of two vector spaces $V$ and $W$ abstractly to be the quotient of the free vector space generated by the symbols $\bov\otimes \bow$, for all $\bov\in V$ and $\bow\in W$, by the relations. We note that if $\hcal$ and $\kcal$ are inner product spaces we may define an inner product on $\hcal\otimes\kcal$ by 
\begin{align}
  \langle \phi_1\otimes \psi_1 | \phi_2\otimes \psi_2\rangle = \langle \phi_1 | \phi_2\rangle \langle \psi_1 | \psi_2\rangle,
\end{align}
and extending linearly to the whole space. If $V$ and $W$ are Hilbert spaces then $V\otimes W$ is easily shown to be a Hilbert space. As we shall see later, once we have defined the trace of an operator, there is relatively explicit construction of the tensor product of Hilbert spaces in terms of the Hilbert-Schmidt class of operators.

If $A:\hcal\to\kcal$ is a linear operator, defined on a dense domain $D(A)\subseteq \hcal$ we may define the domain of the \emph{adjoint} $D(A^*)$, to be the set of vectors $\psi \in \kcal$ for which there exists an $\eta_{\psi}\in \hcal$ such that
\begin{align}
  \langle \psi | A\phi \rangle  = \langle \eta_{\psi}| \phi\rangle,
\end{align}
holds for all $\phi\in D(A)$. We then define $A^*$ on this domain to be
\begin{align}
  A^* \psi = \eta_{\psi},
\end{align}
noting that the density of $D(A)$ ensures that the $\eta_{\psi}$ is unique, for those $\psi$ for which it exists. For $\bops[\hcal]$ the situation is somewhat simpler, the domain of $A$ and $A^*$ can be taken to be $\hcal$, the map $A\mapsto A^*$ is a conjugate linear isomorphism preserving the operator norm \eqref{eqn:operator-norm}, and the equations
\begin{align}
  (AB)^* &= B^* A^*\\
  A^{**} &= A\\
  \left(A^{-1}\right)^* &= \left(A^*\right)^{-1},
\end{align}
hold, with the last equation requiring the additional assumption that $A$ is invertable with bounded inverse. Certain operators mapping a Hilbert space to itself are equal to their own adjoint, we call those the self-adjoint, denoted $\saops[\hcal]$, and note that, where the domains match, they are a vector space over the reals, with pointwise addition and scalar multiplication. 

We now give several versions of the \emph{spectral theorem} a (if not the) central result in functional analysis. We point the reader towards \coma{cite Simom \& Reed} for more information.
\begin{thm}[Spectral theorem for self-adjoint operators - bounded functional calculus form]\label{thm:spectral-sa-ops-bounded-functions}
    Let $A$ be a self-adjoint operator on $\hcal$, there is a unique map $\hat{\phi}$ from the bounded Borel functions on $\R$ to $\bops{\hcal}$ such that
    \begin{itemize}
      \item $\hat{\phi}$ is an algebraic $*$-homomorphism 
      \item $\hat{\phi}$ is norm continuous $\norm{\hat{\phi}(h)} \leq \norm{h}_\infty$
      \item If $h_n$ is a sequence of bounded Borel functions converging (pointwise) to the identity function and $\abs{h_n(x)} \leq x$ for all $x$ and $n$, then for any $\psi\in D(A)$ we have $\lim_{n\to\infty} \hat{\phi}(h_n)\psi = A\psi$
      \item If $h_n \to h$ pointwise and the sequence $\norm{h_n}_\infty$ is bounded then $\hat{\phi}(h_n) \to \hat{\phi}(h)$ strongly
      \item If $A\psi$ = $\lambda\psi$ then $\hat{\phi}(h)\psi = h(\lambda)\psi$
      \item If $h\geq 0$ then $\hat{\phi}(h) \geq 0$
    \end{itemize}
\end{thm}

We use this as a stepping stone to a more general formulation

\begin{thm}[Spectral theorem for self-adjoint operators - Borel functional calculus form]\label{thm:spectral-sa-ops-unbounded-functions}
  Let $A$ be a self-adjoint operator on $\hcal$, and $\chi_\Omega$ the characteristic function of the measurable set $\Omega\subseteq\R$. We define the operators $P_\Omega = \hat{\phi}(\chi_\Omega)$, and note the following properties
  \begin{itemize}
    \item The $P_\Omega$ are orthogonal projections
    \item $P_\R = I$ and $P_\varnothing = 0$
    \item If $\Omega$ is a countable union of disjoint sets $\Omega_n$ then $P_\Omega = \lim_{N\to\infty} \sum_{n=1}^N P_{\Omega_n}$, where the limit converges strongly
    \item $P_\Omega P_\Delta = P_{\Omega\cap\Delta}$.
  \end{itemize}
  The map $\Omega\mapsto\langle\psi| P_\Omega \phi\rangle$ is a complex-valued Borel measure on $\R$ for each $\psi, \phi\in\hcal$ and denote it $\mu_{\psi\phi}$. If $g$ is a bounded Borel function on $\R$ then we can define $g(A)$ by
  \begin{align}
    \langle \psi|g(A)\phi\rangle = \int_\R g(\lambda)\,d\mu_{\psi\phi}(\lambda),
  \end{align}
  and note that this agrees with $\hat{\phi}(g)$, further if $g$ is an unbounded, complex valued, Borel function on $\R$ we can define 
  \begin{align}
    D(g(A)) &= \sbuild{\psi}{\int_\R \abs{g(\lambda)}^2 d\mu_{\psi\psi}(\lambda)<\infty}\\
    \langle \psi | g(A)\phi\rangle &= \int_\R g(\lambda) d\mu_{\psi\phi}(\lambda),
  \end{align}
  We write $g(A) = \int_\R g(\lambda) P_\lambda$, and note that $g(A)$ is self-adjoint if $g$ is real. 
\end{thm}
We call the support of a self-adjoint operator $A$, $\supp{A}$, the complement of the union of all the open sets $\Omega$ for which $P_\Omega = 0$, and note that we can restrict the integrals over $\R$ in the previous theorem to over $\supp{A}$, we therefore generalise to complex-valued Borel functions defined on $\supp{A}$, rather than requiring that they are defined on the whole real line.

Those operators which may be written in the form $A = B^*B$, for some linear operator $B$ are called positive, and the set of such operators forms a \emph{convex  cone} \coma{ref to convex section} denoted $\posops[\hcal]$. For any $\psi\in D(A^*A)$ we note that 
\begin{align}
  \langle\psi | A* A \psi \rangle &= \langle A \psi | A \psi\rangle \\
                                  &= \norm{A\psi}^2 \geq 0.
\end{align}

If $\seq{\boe_k}[k\in\N]$ is an orthonormal basis for a Hilbert space $\hcal$ then we define the trace of an operator $A\in\bops[\hcal]$ 
\begin{align}
  \tr{A} &= \sum_k \braket{\boe_k}{A\boe_k},
\end{align}
noting that \coma{cite?} wherever this sum converges for one choice of $\seq{\boe_k}[k\in\N]$ its value is independent of the choice of basis. We call the operators for which the trace is defined the trace-class and denote them $\trops[\hcal]$, possibly omitting the argument where the underlying Hilbert space may be assumed. The trace-class operators are a subset of the bounded operators, closed under (pointwise) addition and scalar multiplication, and 
\section{Quantum theory}

\subsection{Notation and conventions}

Throughout this thesis we will use the standard formulation of quantum mechanics. All Hilbert spaces may be assumed to be over the field of complex numbers. The bounded linear operators on a Hilbert space $\hcal$ will be denoted $\bops[\hcal]$, the self-adjoint operators $\saops[\hcal]$, and the positive cone $\posops[\hcal]$. The trace-class will be denoted $\trops[\hcal]$. We will use $\dops[\hcal]$ for the states of quantum mechanics, the positive operators with trace-$1$. Where the underlying Hilbert space has been fixed we may omit it. 

For the observables of a quantum system we use the formalism of positive operator valued measures (henceforth POVMs). In this setting an observable $\ope$ with outcome set $\Omega$ on a system described by Hilbert space $\hcal$ is a function $\ope:\acal\to\posops[\hcal]$, where $\acal$ is a sigma algebra of subsets of $\Omega$, called the event space. In addition we require that that $\ope$ is normalised so that $\ope(\Omega) = \opi$ and $\sigma$-additive in the sense that $\ope\left(\bigcup_{n=1}^\infty X_n\right) = \sum_{n=1}^\infty \ope(X_n)$, where the $X_n$ are a sequence of disjoint sets, and  the series converges weakly, or equivalently strongly. We will see in proposition \ref{prop:povms-most-general-obs} that these POVMs are, in fact, the most general way of defining observables which respect the convex structure of $\dops$. Where the outcome set $\Omega$ is finite, and the set of events $\acal$ is the entire power set of $\Omega$ a simpler definition suffices, in this case a POVM is defined entirely by its action on the singleton sets, so we can equivalently consider a map $\ope:\Omega\to\posops$, such that $\sum_{\omega\in\Omega} \ope(\omega) = \opi$. We also define the probability measure $\ope_\rho:X\mapsto \tr{\ope(X)\rho}$ where $\ope$ is a POVM and $\rho$ is a state.

For the most part we will deal only with probability spaces where the outcome set is $\Omega\subseteq\R$, in this case we can define the variance of a probability measure $\mu$
\begin{align}
  \var{\mu} \defeq \inf_{x_0\in\R}\int_\Omega (x-x_0)^2 d\mu(x),
\end{align}
similarly where $\ope$ is a POVM with outcome set a subset of the reals we define
\begin{align}
  \var[\rho]{\ope} \defeq \var{\ope_\rho}.
\end{align}
Where it makes the text simpler we may use a self-adjoint operator in place of a POVM in such formulae, in these cases we take the measure to be the spectral measure of the operator.


\subsection{Observables}

\begin{prop}\label{prop:povms-most-general-obs}
  Let $\Omega$ be an outcome set, $\acal$ a sigma algebra on $\Omega$, $\hcal$ a Hilbert space and $\probs{\Omega}{\acal}$ the space of probability measures on $(\Omega, \acal)$. Any map $M:\dops\to\probs{\Omega}{\acal}$, for which 
  \begin{equation}\label{eq:obs-map-linear}
    M(\rho) = \sum_i w_i M(\rho_i),
  \end{equation}
  for all convex decompositions $\sum_i w_i \rho_i$, of all states $\rho\in\dops[\hcal]$ may be represented as
  \begin{equation}
    M(\rho)(X) = \tr{\rho\ope(X)},
  \end{equation}
  where $\ope$ is POVM, fixed by the choice of $M$.
\end{prop}

\begin{proof}
  We begin by noting that for any fixed $X\in\acal$ the map $\rho\mapsto M(\rho)(X)$ extends to a linear functional on $\trops[\hcal]$, by the well known duality $\trops[\hcal]^* \simeq \bops[\hcal]$ there is an element $\ope(X)\in\bops[\hcal]$ such that $M(\rho)(X) = \tr{\rho\ope(X)}$, for all $\rho\in\dops[\hcal]$. Varying $\rho$ we see that $\ope(X)$ is positive and bounded by the identity $\opi$, now varying $X$ we see that $\ope(\Omega) = \opi$, and the $\ope$ is $\sigma$-additive, and is therefore a POVM.

\end{proof}

\section{Convex analysis}

Convex analysis concerns itself with convex sets and convex functions. Although there is an interesting theory of convex subsets of infinite dimensional vector spaces \coma{cite?} this is beyond the scope of this thesis. Here all convex sets will be subsets of finite dimensional, real vector spaces. We will follow the exposition of Rockerfellar \cite{rtr-conv-anal-book}. We write as $V^*$ the topological dual space of a topological vector space $V$, the space of continuous linear functionals on $V$. Since we are only dealing with finite dimensional spaces a vector space $V$ is canonically isometrically isomorphic to its bidual $V^{**}$, and we do not distinguish the two, equating the vector $\boex\in V$ and the evaluation map $(\psi\mapsto\psi(\boex))\in V^{**}$. This identification is convenient for discussing the convex conjugate of a function.

\begin{defn}\label{defn:convex-set}
  A subset $C$ of a real vector space is \emph{convex} if for all $\boex,\boy\in C$ and for all $\lambda\in [0,1]$
  \begin{align}
    \lambda \boex + (1-\lambda)\boy\in C.
  \end{align}
\end{defn}
Examples of convex sets include
\begin{itemize}
  \item Any interval in the reals (viewed as a one dimensional vector space over themselves)
  \item The balls in any normed vector space over the reals
  \item The state-space $\dops$ of quantum mechanics
  \item The set of points in $\R^2$ that lie ``above'' the graph of the function $x\mapsto x^2$, i.e. the set of points $\sbuild{(x,y)}{y > x^2}$.
\end{itemize}
This last example motivates the definition of a convex function
\begin{defn}\label{defn:epigraph}
  The \emph{epigraph} of a function $f:V\to\R$, where $V$ is a real vector space is the set of points in the vector space $V\oplus \R$ \coma{is direct sum correct here??} ``above'' the graph of $f$
  \begin{align}
    \epi f = \sbuild{(v, \mu)}{v\in V,\, \mu\in\R, \mu \geq f(v)}.
  \end{align}
\end{defn}
\begin{defn}\label{defn:real-convex-function}
  A function from a real vector-space to the reals is \emph{convex} if its epigraph is a convex set.
\end{defn}
This definition of convexity for functions is slightly too restrictive, because it only covers functions defined on an entire vector space $V$. We will mainly be concerned with uncertainty measures which are non-negative, and so we are motivated to find a definition of convexity appropriate for more general domains.
\begin{defn}\label{defn:extended reals}
  We define the \emph{extended reals} as an extension of the ordered field of real numbers $\exR = \R \cup \{\infty, -\infty\}$ with the following axioms
  \begin{enumerate}
    \item $\forall x\in\R$, $\infty > x$ and $-\infty < x$,
    \item $\forall x\in\R$, $x + \infty = \infty + x = \infty$,
    \item $\forall x\in\R$, $x + \infty = \infty + x = \infty$,
    \item $\infty + \infty = \infty$ and $-\infty - \infty = -\infty$,
    \item $\forall x\in\R$, $x>0 \implies x\infty = \infty x =  \infty$ and $x(-\infty) = (-\infty)x = -\infty$,
    \item $\forall x\in\R$, $x<0 \implies x\infty = \infty x = -\infty$ and $x(-\infty) = (-\infty)x = \infty$,
    \item $0 \infty = \infty 0 = 0 = 0(-\infty) = (-\infty)0$,
    \item $\inf\varnothing = \infty$ and $\sup\varnothing = -\infty$.
  \end{enumerate}
\end{defn}
We can now consider functions defined on arbitrary convex subsets of real vector spaces, simply by extending the function to be $\infty$ outside. We note that strictly this recourse to the extended reals is unnecessary and one can consider convex functions defined on convex subsets of real vector spaces directly, but this leads to more tedious consideration of domains. In the present formulation one can recover the domain by considering the points where the extended function takes finite values. It is also important to note that the forms $\infty-\infty$ and $-\infty +\infty$ are left undefined. As Rockerfellar notes, it is in principle important to be cautious when these cases arise, but in practice they occur very rarely.

For completeness we define the epigraph and convexity for extended real functions, although these are essentially identical to definitions given above.
\begin{defn}\label{defn:extended-convex-function}
  A function $f:V\to\exR$, where $V$ is a real vector space, is \emph{convex} if its epigraph
  \begin{align}
    \epi f = \sbuild{(v,\mu)}{v\in V,\,\mu\in \R,\,\mu\geq f(v)},
  \end{align}
  is a convex subset of $V\oplus R$. Note that if $f(v) =\infty$ there are no $\mu\in\R$ such that $\mu\geq f(v)$, so these $v$ do not occur in any of the ordered pairs in $\epi f$.
\end{defn}
We now state two theorems useful theorems of convex functions
\begin{thm}\label{thm:jensen-two-point}
  Let $f:V\to (-\infty, \infty]$, with $V$ a real vector space. Then $f$ is convex if and only if the inequality
  \begin{align}
    f(\lambda \boex + (1-\lambda)\boy) \leq \lambda f(\boex) + (1-\lambda)f(\boy),
  \end{align}
  holds for all $\boex,\boy\in V$ and $\lambda\in [0,1]$. The exception of $-\infty$ is to exclude pathological cases where $f(\boex)$ and $f(\boy)$ are different infinite values.
\end{thm}
\begin{thm}[Jensen's inequality]\label{thm:jensen-n-point}
  Let $f:V\to (-\infty, \infty]$, with $V$ a real vector space. Then $f$ is convex if and only if the inequality
  \begin{align}
    f\left( \sum_{i=1}^n \lambda_i \boex_i \right) \leq \sum_{i=1}^n\lambda_i f(\boex_i),
  \end{align}
  holds for all $\boex_i\in V$ and $\lambda_i\in [0,1]$ such that $\sum_{i=1}^{n}\lambda_i = 1$.
\end{thm}
There are also generalisations to measurable sets of points, rather than finite numbers. 

It is useful to define the \emph{convex conjugate} of a function, we will apply this extensively in section \coma{put something here when you've written the section}

\begin{defn}\label{defn:convex conjugate}
  Given a function $f:V\to\exR$, where $V$ is a topological vector space over the reals, we define the convex conjugate of $f$ to be
  \begin{align}
    f^*&:V^*\to \exR\\
    f^*&:\boalpha\mapsto\sup_{\bov\in V} \{\langle\boalpha,\bov\rangle - f(\bov)\},
  \end{align}
  where, $V^*$ is the space of continuous linear functionals on $V$ and $\langle\cdot,\cdot\rangle$ denotes the dual pairing between $V$ and $V^*$. 
\end{defn}

\begin{thm}
  The biconjugate $f^{**}$ of a function $f:V\to\exR$ is greatest (pointwise) lower-semi continuous function which is bounded above by $f$.
\end{thm}

Where $f$ is a convex function $(f^*)^*$ (hereafter denoted $f^{**}$) is equal to $f$.

\section{Semidefinite programming}

The usual formulation \coma{e.g. cite} of semi-definite programming is in terms of $n\times n$ positive-semidefinite real matrices with real elements. Here, following the exposition given in \cite{w-semidefinite-prog-cb-norms}, we use a different, but entirely equivalent, formulation better adapted to problems in quantum mechanics. 

\begin{defn}\label{defn:complex-semidefinite-program}
  Let $\hcal$, $\kcal$ be finite dimensional Hilbert spaces, $\opc\in\saops[\hcal]$, $\opd\in\saops[\kcal]$ and let $\Psi:\saops[\hcal]\to\saops[\kcal]$ be a linear map. The \emph{primal semidefinite problem} and \emph{dual semidefinite problem}  associated to the triple $\left(\Psi, \opc, \opd\right)$ are
  \begin{equation}
  \begin{aligned}
    & \underset{X\in\posops[\hcal]}{\text{maximise}} & &\tr{\opc X } \qquad\qquad & & \underset{Y\in\posops[\kcal]}{\text{minimise}} & & \tr{\opd Y} \\
    & \text{subject to} & &\Psi(X) \leq \opd & & \text{subject to} & &  \Psi^*(Y) \geq C,
  \end{aligned}
  \label{eqn:semidefinite-program}
\end{equation}
respectively.
\end{defn}

Here it is traditional to note the analogy with the classical theory of linear programming,  we therefore note the duality theorem for linear programs \eqref{thm:duality-linear-prog} from ref.~\cite{schrijver-lin-int-prog}.
\begin{thm}\label{thm:duality-linear-prog}
  Let $\opa\in\R^{n\times m}$, $\boc\in\R^n$ and $\bob\in\R^m$ then
  \begin{align}
    \sup\left\{\boc\cdot \boex \middle| \boex\in\R^n,\, A \boex \leq \bob\right \} = \inf\left\{ \bob\cdot\boy \middle|\boy\in\R^m,\, A^T \boy= \boc\right \},
  \end{align}
  if at least one of the sets is non-empty, with the convention that the $\sup$ and $\inf$ of an empty set are $-\infty$ and $\infty$ respectively.
\end{thm}
  
The analogy follows from considering the trace of the product of two operators as an inner product on the (real) vector vector space of self-adjoint operators on a given Hilbert space. The extra complication that comes from considering operator inequalities rather than vector inequalities causes the duality theory for semidefinite programs to be somewhat weaker than that for linear programs.

\begin{defn}\label{defn:semidefinite-feasible-sets}
  Given a triple $\left(\Psi, \opc, \opd\right)$, chosen as in definition \ref{defn:complex-semidefinite-program} we define the \emph{primal feasible set} and \emph{dual feasible set} to be
  \begin{align}
    \pcal &= \sbuild{X\in\posops[\hcal]}{\Psi(X)\leq \opd},
  \end{align}
  and
  \begin{align}
    \dcal &= \sbuild{Y\in\posops[\kcal]}{\Psi^*(Y)\geq \opc},
  \end{align}
  respectively.
\end{defn}

\begin{thm}[Weak duality]\label{thm:weak-duality-semidefinite-prog}
  For every triple $\left(\Psi, \opc, \opd\right)$ chosen as in definition \ref{defn:complex-semidefinite-program} the inequality
  \begin{align}
    \sup_{X\in\pcal} \tr{\opc X} \leq \inf_{Y\in\dcal} \tr{\opd Y},
  \end{align}
  holds. A proof is contained in ref.~\cite{Vandenberghe-Boyd-semidefinite}.
\end{thm}

We call a semidefinite problem \emph{strongly dual} if we have
\begin{align}\label{eqn:strong-duality-equality}
  \sup_{X\in\pcal} \tr{\opc X} = \inf_{Y\in\dcal} \tr{\opd Y},
\end{align}
although necessary conditions are not easy to find Slater's condition \coma{cite} is sufficient to prove weak duality
\begin{thm}{Slater's condition}\label{eqn:slaters-condition-sufficient}
  Let $\left(\Psi, \opc, \opd\right)$ be chosen as in definition \ref{defn:complex-semidefinite-program} then the following two implications are true
  \begin{enumerate}
  \item If $\inf_{Y\in\dcal} \tr{\opd Y} \in\R$ and there exists an operator $X> 0$ such that $\Psi(X) < \opd$, then the equality \eqref{eqn:strong-duality-equality} holds and there exists an operator $Y\in\dcal$ achieving the infimum.
  \item If $\sup_{Y\in\pcal} \tr{\opc X} \in\R$ and there exists an operator $Y> 0$ such that $\Psi^*(Y) > \opc$, then the equality \eqref{eqn:strong-duality-equality} holds and there exists an operator $X\in\pcal$ achieving the supremum.
  \end{enumerate}
\end{thm}

