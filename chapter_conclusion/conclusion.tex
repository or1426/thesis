\let\textcircled=\pgftextcircled
\chapter{Conclusions and Outlook} \label{chap:conclusion}


In part I of this thesis we examined preparation uncertainty in both finite and infinite dimensional contexts. We give a new, geometric, derivation of the uncertainty region for sharp $\pm 1$, valued observables. We also provide simple counterexamples demonstrating that the Schr{\"o}dinger uncertainty relation does not always suffice to fully characterise the uncertainty region for Hilbert spaces of dimension greater than $2$. We also compute the uncertainty region for a pair of qutrit observables arising as spectral measures of non-commuting operators, which commute on a subspace, and show that it contains the point where both probability measures are deterministic.

We also examine the preparation uncertainty for the position and momentum observables arising from the well known example of the ``particle in a box'' system. We compare this with the preparation uncertainty of the free particle, as well as the particle on a ring, previously examined in ref.~\cite{sharp-ur-num-angle}. The box system lacks the full phase space symmetry which makes the analysis possible for the ring and free particle. Consequently we can not use the methods previously employed to analyse the uncertainty region. Nonetheless we obtain an upper bound on the boundary of the uncertainty and show it is exact in some interval. Explicitly it is exact for position uncertainties greater than some bound. We also show that the well-known canonical hyperbola, attributed to Heisenberg, is a lower bound for the boundary of the uncertainty region and that the difference between the upper and lower bound is small\footnote{At the point where they are furthest apart the upper bound is $\frac{1}{2}$ and the lower bound is $\frac{1}{2\sqrt{\frac{\pi^2}{3} -2}}\approx 0.44$}, and converges to zero as the position uncertainty becomes small. This reflects the physical intuition that as the states become highly concentrated in position the influence of the boundary conditions becomes less.

An obvious open question is to characterise the uncertainty region in the interval where we only have upper and lower bounds. We conjecture that the upper bound shown in figure~\ref{fig:box-ur} is exact globally, rather than just in the region where we have shown it is exact. More strongly, it seems natural to assume that among the states which minimise the momentum uncertainty for a fixed position uncertainty, there are those with position expectation zero. We began analysing the box system as part of a study of preparation uncertainty in the context of multi-slit interferometry. In particular we believe the particle in a box and particle on a ring systems arise if one imposes certain covariance conditions on the which-way and fringe-contrast observables of an idealised infinite interferometer. It would be interesting to explore this further, in particular the physical interpretation of the covariance conditions are not clear. If this can be achieved it might also be interesting to explore alternative covariance conditions.

Little is known about measurement uncertainty in infinite dimensional systems in the absence of the full phase-space symmetry. One could also investigate what could be achieved exploiting only the partial phase-space symmetry exhibited by the ``particle in a box'' setup.

%=========================================================

In part II we provide a systematic framework for exploiting the covariance properties of a family of observables to determine their measurement uncertainty region. This framework may be applied to arbitrary families of finite-outcome observables and a wide class of natural distance measures for observables. Specifically we define a map we call the ``invariant mean'' which acts as a projection on the space of bounded operator-valued functions on the outcome set. We show that for jointly convex error measures, ``compatible'' with the symmetry group action, the best compatible approximations to covariant observables are also covariant.

We apply these methods to the case of a metric on observables based on the $p$-norm, and completely characterise the measurement uncertainty region for a triple of qubit Pauli observables, and for the quantum Fourier pair in arbitrary finite dimensions. Unknown to us the latter case was addressed already in~\cite{Werner2016}, although our method of proof is different and provides a case study of the use of the invariant mean in solving measurement uncertainty problems.

The jointly convex error measures that our framework applies to include the $f$-divergences introduced in \cite{10.2307/2984279} and \cite{der1964informationstheoretische}, however the $f$-divergences may be infinite unless the approximating probability distribution is dominated by the target. If we attempt to follow the previous work and define an $f$-divergence for quantum observables by taking the supremum over quantum states we will, in general, break this constraint. It might be interesting to characterise the class of functions $f$ for which this does not occur. A reasonable starting point would be those $f$ for which the $\lim_{f\to\infty}f(x) \in \R$, including $f:t\mapsto \frac{1}{t}-1$, which defines the Neyman divergence.

It would be useful to generalise the invariant mean construction to positive operator valued measures, with infinite outcome sets, for example the canonical position and momentum for a quantum particle. In this case it is not generally possible to express the observable as a simple function from the outcome set. Is not known if the sum over group elements we use to define the invariant mean map may be replaced by a Bochner integral with respect to the Haar measure. Even if the integral is well defined, the resulting quantity may not be an observable.