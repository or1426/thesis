
%UoY:

%The abstract shall follow the title page. It shall provide a synopsis of the thesis, stating the nature and scope of work undertaken and the contribution made to knowledge in the subject treated. It shall appear on its own on a single page and shall not exceed 300 words in length. The abstract of the thesis may, after the award of the degree, be published by the University in any manner approved by the Senate, and for this purpose, the copyright of the abstract shall be deemed to be vested in the University.


\chapter*{Abstract}
\addcontentsline{toc}{chapter}{Abstract}
%\begin{SingleSpace}
This thesis addresses two forms of quantum uncertainty. In part \ref{part:prep-ur}, we focus on preparation uncertainty, an expression of the fact that there are sets of observables for which the induced probability distributions are not simultaneously sharp in any state. We exactly characterise the preparation uncertainty regions for several finite dimensional case studies, including a new derivation of the preparation uncertainty region for the Pauli observables of qubits, and two qutrit case studies which have not previously been addressed in the literature.\\
We also consider the variance based preparation uncertainty for position and momentum observables for the well known ``particle in  a box'' system. We see that the appropriate momentum observable is \emph{not} given by the spectral measure of a self-adjoint operator, although the position observable is. The box system lacks the phase-space symmetry used to determine the free particle and particle on a ring systems so determining the box uncertainty region is rather more difficult than in these cases. We give upper and lower bounds on the boundary of the uncertainty region, and show that our upper bound is exact in an interval.\\
In part \ref{part:meas-ur} we turn our attention to measurement uncertainty, exploring the space of compatible joint approximations to incompatible target observables. We prove a general theorem, which shows that, for a broad class of figures of merit, the optimal compatible approximations to covariant targets are themselves covariant. This substantially simplifies the problem of determining measurement uncertainty regions for covariant observables, since the space of covariant compatible approximations is smaller than the space of all compatible approximations.\\
We employ this theorem to derive measurement uncertainty regions for three mutually orthogonal Pauli observables, and for the quantum Fourier pair acting in any finite dimension.
%\end{SingleSpace}
\clearpage