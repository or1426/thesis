\RequirePackage[l2tabu]{nag}		% Warns for incorrect (obsolete) LaTeX usage
%
% Based on the following template, adapted by Gaurav Dhariwal and Mirjam Weilenmann to match the UoY guidelines.
% File: memoirthesis.tex
% Author: Victor Brena
% Description: Contains the thesis template using memoir class,
% which is mainly based on book class but permits better control of 
% chapter styles for example. This template is an adaptation and 
% modification of Oscar's.
% 
% Memoir is a flexible class for typesetting poetry, fiction, 
% non-fiction and mathematical works as books, reports, articles or
% manuscripts. CTAN repository is found at:
% http://www.ctan.org/tex-archive/macros/latex/contrib/memoir/
%
% Memoir class loads useful packages by default (see manual).
\documentclass[a4paper,11pt,leqno,openbib]{memoir} %add 'draft' to turn draft option on (see below)
%
%
% Adding metadata:
\usepackage{datetime}
\usepackage{ifpdf}
\ifpdf
\pdfinfo{
   /Author Oliver Owen Douglas Reardon-Smith
   /Title (PhD Thesis)
   /Keywords (One; Two; Three)
   /CreationDate (D:\pdfdate)
}
\fi
% When draft option is on. 
\ifdraftdoc 
	\usepackage{draftwatermark}				%Sets watermarks up.
	\SetWatermarkScale{0.3}
	\SetWatermarkText{\textbf Draft: \today}
\fi
%
% Declare figure/table as a subfloat.
\newsubfloat{figure}
\newsubfloat{table}
% Better page layout for A4 paper, see memoir manual.
\settrimmedsize{297mm}{210mm}{*}
\setlength{\trimtop}{0pt} 
\setlength{\trimedge}{\stockwidth} 
\addtolength{\trimedge}{-\paperwidth} 
\settypeblocksize{634pt}{448.13pt}{*} 
\setulmargins{4cm}{*}{*} 
\setlrmargins{*}{*}{1.5} 
\setmarginnotes{17pt}{51pt}{\onelineskip} 
\setheadfoot{\onelineskip}{2\onelineskip} 
\setheaderspaces{*}{2\onelineskip}{*} 
\checkandfixthelayout
%

\usepackage{fouriernc}
\usepackage[T1]{fontenc}

\frenchspacing

%
\OnehalfSpacing 
%
% Sets numbering division level
\setsecnumdepth{subsection} 
\maxsecnumdepth{subsubsection}
%
% Chapter style (taken and slightly modified from Lars Madsen Memoir Chapter 
% Styles document
\usepackage{calc,soul,fourier}
\makeatletter 
\newlength\dlf@normtxtw 
\setlength\dlf@normtxtw{\textwidth} 
\newsavebox{\feline@chapter} 
\newcommand\feline@chapter@marker[1][4cm]{%
	\sbox\feline@chapter{% 
		\resizebox{!}{#1}{\fboxsep=1pt%
			\colorbox{black}{\color{white}\thechapter}% 
		}}%
		\rotatebox{90}{% 
			\resizebox{%
				\heightof{\usebox{\feline@chapter}}+\depthof{\usebox{\feline@chapter}}}% 
			{!}{\scshape\so\@chapapp}}\quad%
		\raisebox{\depthof{\usebox{\feline@chapter}}}{\usebox{\feline@chapter}}%
} 
\newcommand\feline@chm[1][4cm]{%
	\sbox\feline@chapter{\feline@chapter@marker[#1]}% 
	\makebox[0pt][c]{% aka \rlap
		\makebox[1cm][r]{\usebox\feline@chapter}%
	}}
\makechapterstyle{daleifmodif}{
	\renewcommand\chapnamefont{\normalfont\Large\scshape\raggedleft\so} 
	\renewcommand\chaptitlefont{\normalfont\Large\bfseries\scshape} 
	\renewcommand\chapternamenum{} \renewcommand\printchaptername{} 
	\renewcommand\printchapternum{\null\hfill\feline@chm[2.5cm]\par} 
	\renewcommand\afterchapternum{\par\vskip\midchapskip} 
	\renewcommand\printchaptertitle[1]{\color{black}\chaptitlefont\raggedright ##1\par}
} 
\makeatother 

\chapterstyle{daleifmodif}
%
\makepagestyle{myvf} 
\makeoddfoot{myvf}{}{\thepage}{}
\makeevenfoot{myvf}{}{\thepage}{} 
\makeheadrule{myvf}{\textwidth}{\normalrulethickness} 
\makeevenhead{myvf}{\tiny\textsc{\leftmark}}{}{} 
\makeoddhead{myvf}{}{}{\tiny\textsc{\rightmark}}
\pagestyle{myvf}
%
% Oscar's command (it works):
% Fills blank pages until next odd-numbered page. Used to emulate single-sided
% frontmatter. This will work for title, abstract and declaration. Though the
% contents sections will each start on an odd-numbered page they will
% spill over onto the even-numbered pages if extending beyond one page
% (hopefully, this is ok).
\newcommand{\clearemptydoublepage}{\newpage{\thispagestyle{empty}\cleardoublepage}}
%
%
% Creates indexes for Table of Contents, List of Figures, List of Tables and Index
\makeindex
% 
\usepackage{import}

\usepackage{color}

\usepackage[normalem]{ulem}


% Add other packages needed for chapters here. For example:
\usepackage{lipsum}					%Needed to create dummy text
\usepackage{amsfonts} 					%Calls Amer. Math. Soc. (AMS) fonts
\usepackage[centertags]{amsmath}			%Writes maths centred down
\usepackage{stmaryrd}					%New AMS symbols
\usepackage{amssymb}					%Calls AMS symbols
\usepackage{amsthm}					%Calls AMS theorem environment
\usepackage{newlfont}					%Helpful package for fonts and symbols
\usepackage{layouts}					%Layout diagrams
\usepackage{graphicx}					%Calls figure environment
\usepackage{longtable,rotating}			%Long tab environments including rotation. 
\usepackage[utf8]{inputenc}			%Needed to encode non-english characters 
%Changed from applemac to utf8 (mirjam)
					
%directly for mac
\usepackage{dsfont}									
\usepackage{colortbl}					%Makes coloured tables
\usepackage{wasysym}					%More math symbols
\usepackage{mathrsfs}					%Even more math symbols
\usepackage{float}						%Helps to place figures, tables, etc. 
\usepackage{verbatim}					%Permits pre-formated text insertion
\usepackage{upgreek}					%Calls other kind of greek alphabet
\usepackage{latexsym}					%Extra symbols
\usepackage[square,numbers,
		     sort&compress]{natbib}		%Calls bibliography commands 
\usepackage{url}						%Supports url commands
\usepackage{etex}						%eTeXÕs extended support for counters
%\usepackage{fixltx2e}					%Eliminates some in felicities of the 
									%original LaTeX kernel
\usepackage[spanish,main=english,german,french]{babel}		%For languages characters and hyphenation
\usepackage{color}                    				%Creates coloured text and background
\usepackage[colorlinks=true,
		     allcolors=black]{hyperref}              %Creates hyperlinks in cross references
\usepackage{memhfixc}					%Must be used on memoir document 
									%class after hyperref
\usepackage{enumerate}					%For enumeration counter
\usepackage{footnote}					%For footnotes
\usepackage{microtype}					%Makes pdf look better.
\usepackage{rotfloat}					%For rotating and float environments as tables, 
									%figures, etc. 
\usepackage{alltt}						%LaTeX commands are not disabled in 
									%verbatim-like environment
\usepackage[version=0.96]{pgf}			%PGF/TikZ is a tandem of languages for producing vector graphics from a 
\usepackage{tikz}						%geometric/algebraic description.
\usetikzlibrary{arrows,shapes,snakes,
		       automata,backgrounds,
		       petri,topaths}				%To use diverse features from tikz		
%							
%Reduce widows  (the last line of a paragraph at the start of a page) and orphans 
% (the first line of paragraph at the end of a page)
\widowpenalty=1000
\clubpenalty=1000
%
% New command definitions for my thesis
%
%\DeclareSymbolFont{bbold}{U}{bbold}{m}{n}
%\DeclareSymbolFontAlphabet{\mathbbold}{bbold}
%\newcommand\ind{\mathbbold{1}}
\usepackage[bb=boondox]{mathalfa}

\usepackage{bbm} %to make bbm symbols
\usepackage[makeroom]{cancel}


\newcommand{\keywords}[1]{\par\noindent{\small{\textbf Keywords:} #1}} %Defines keywords small section
\newcommand{\parcial}[2]{\frac{\partial#1}{\partial#2}}                             %Defines a partial operator
\newcommand{\vectorr}[1]{\mathbf{#1}}                                                        %Defines a bold vector
\newcommand{\vecol}[2]{\left(                                                                         %Defines a column vector
	\begin{array}{c} 
		\displaystyle#1 \\
		\displaystyle#2
	\end{array}\right)}
\newcommand{\mados}[4]{\left(                                                                       %Defines a 2x2 matrix
	\begin{array}{cc}
		\displaystyle#1 &\displaystyle #2 \\
		\displaystyle#3 & \displaystyle#4
	\end{array}\right)}
\newcommand{\pgftextcircled}[1]{                                                                    %Defines encircled text
    \setbox0=\hbox{#1}%
    \dimen0\wd0%
    \divide\dimen0 by 2%
    \begin{tikzpicture}[baseline=(a.base)]%
        \useasboundingbox (-\the\dimen0,0pt) rectangle (\the\dimen0,1pt);
        \node[circle,draw,outer sep=0pt,inner sep=0.1ex] (a) {#1};
    \end{tikzpicture}
}
\newcommand{\range}[1]{\textnormal{range }#1}                                             %Defines range operator
\newcommand{\innerp}[2]{\left\langle#1,#2\right\rangle}                                 %Defines inner product
\newcommand{\prom}[1]{\left\langle#1\right\rangle}                                         %Defines average operator
\newcommand{\tra}[1]{\textnormal{tra} \: #1}                                                       %Defines trace operator
\newcommand{\sign}[1]{\textnormal{sign\,}#1}                                                   %Defines sign operator
\newcommand{\sech}[1]{\textnormal{sech} #1}                                                  %Defines sech
\newcommand{\diag}[1]{\textnormal{diag} #1}                                                    %Defines diag operator
\newcommand{\arcsech}[1]{\textnormal{arcsech} #1}                                       %Defines arcsech
\newcommand{\arctanh}[1]{\textnormal{arctanh} #1}                                         %Defines arctanh
%Change tombstone symbol
\newcommand{\blackged}{\hfill$\blacksquare$}
\newcommand{\whiteged}{\hfill$\square$}
\newcounter{proofcount}
\renewenvironment{proof}[1][\proofname.]{\par
 \ifnum \theproofcount>0 \pushQED{\whiteged} \else \pushQED{\blackged} \fi%
 \refstepcounter{proofcount}
 \normalfont 
 \trivlist
 \item[\hskip\labelsep
       \itshape
   {\textbf\em #1}]\ignorespaces
}{%
 \addtocounter{proofcount}{-1}
 \popQED\endtrivlist
}
%
%
% New definition of square root:
% it renames \sqrt as \oldsqrt
\let\oldsqrt\sqrt
% it defines the new \sqrt in terms of the old one
\def\sqrt{\mathpalette\DHLhksqrt}
\def\DHLhksqrt#1#2{%
\setbox0=\hbox{$#1\oldsqrt{#2\,}$}\dimen0=\ht0
\advance\dimen0-0.2\ht0
\setbox2=\hbox{\vrule height\ht0 depth -\dimen0}%
{\box0\lower0.4pt\box2}}
%
% My caption style
\newcommand{\mycaption}[2][\@empty]{
	\captionnamefont{\scshape} 
	\changecaptionwidth
	\captionwidth{0.9\linewidth}
	\captiondelim{.\:} 
	\indentcaption{0.75cm}
	\captionstyle[\centering]{}
	\setlength{\belowcaptionskip}{10pt}
	\ifx \@empty#1 \caption{#2}\else \caption[#1]{#2}
}
%
% My subcaption style
\newcommand{\mysubcaption}[2][\@empty]{
	\subcaptionsize{\small}
	\hangsubcaption
	\subcaptionlabelfont{\rmfamily}
	\sidecapstyle{\raggedright}
	\setlength{\belowcaptionskip}{10pt}
	\ifx \@empty#1 \subcaption{#2}\else \subcaption[#1]{#2}
}
%
%An initial of the very first character of the content
\usepackage{lettrine}
\newcommand{\initial}[1]{%
	\lettrine[lines=3,lhang=0.33,nindent=0em]{
		\color{black}
     		{\textsc{#1}}}{}}
%
% Theorem styles used in my thesis
%
\theoremstyle{plain}
\newtheorem{theo}{Theorem}[chapter]
\theoremstyle{plain}
\newtheorem{prop}{Proposition}[chapter]
\theoremstyle{plain}
\theoremstyle{definition}
\newtheorem{dfn}{Definition}[chapter]
\theoremstyle{plain}
\newtheorem{lema}{Lemma}[chapter]
\theoremstyle{plain}
\newtheorem{cor}{Corollary}[chapter]
\theoremstyle{plain}
\newtheorem{resu}{Result}[chapter]
%
% Hyphenation for some words
%
\hyphenation{res-pec-tively}
\hyphenation{mono-ti-ca-lly}
\hyphenation{hypo-the-sis}
\hyphenation{para-me-ters}
\hyphenation{sol-va-bi-li-ty}

\numberwithin{equation}{section}



%
%
\begin{document}
%
\renewcommand{\labelitemi}{$\bullet$}

\renewcommand{\qedsymbol}{$\blacksquare$}



\newcommand{\indep}{\rotatebox[origin=c]{90}{$\models$}}
\newcommand{\ot}{\otimes}
\newcommand{\eps}{\epsilon}
\newcommand{\sth}{ \ \mathrm{s.t.} \ }
\newcommand{\trans}{\mathrm{T}}
\newcommand{\tr}[1]{\operatorname{tr}\left( #1 \right)}
\newcommand{\partr}[2]{\operatorname{tr}_{#2}\left( #1 \right)}
\newcommand{\trdist}[2]{\operatorname{D}\left( #1 | #2 \right)}
\newcommand{\lin}[1]{\mathcal{L}\left( #1 \right)} %linear operators...
\newcommand{\linH}[1]{\mathcal{L}_\mathrm{H}\left( #1 \right)}

%Quantum 
\newcommand{\ket}[1]{| #1 \rangle}
\newcommand{\bra}[1]{\langle #1 |}
\newcommand{\braket}[2]{\langle #1|#2\rangle}
\newcommand{\ketbra}[2]{|#1\rangle\!\langle#2|}
\newcommand{\proj}[1]{\ketbra{#1}{#1}}
\newcommand{\identity}{\openone}
\newcommand{\id}{\mathbbm{1}}
\newcommand{\cptp}{\mathcal{E}}


\newcommand{\bcal}{\mathcal{B}}
\newcommand{\ccal}{\mathcal{C}}
\newcommand{\ecal}{\mathcal{E}}
\newcommand{\fcal}{\mathcal{F}}
\newcommand{\lcal}{\mathcal{L}}
\newcommand{\ocal}{\mathcal{O}}
\newcommand{\vcal}{\mathcal{V}}
\newcommand{\zcal}{\mathcal{Z}}
\newcommand{\kcal}{\mathcal{K}}

\newcommand{\fmath}{\mathbb{F}}
\newcommand{\smath}{\mathbb{S}}
\newcommand{\norm}[3]{\vert  #1 {\vert }_{#2}^{#3}} %%26/12/2013
\newcommand{\Norm}[3]{\Vert  #1 {\Vert }_{#2}^{#3}} %%26/12/2013
\
%\newcommand{\eps}{\varepsilon}
\newcommand{\then}{\Rightarrow}
\newcommand\Eb{\mathbb{E}}
\newcommand\N{\mathbb{N}}
\newcommand\Z{\mathbb{Z}}
\newcommand\Q{\mathbb{Q}}
\newcommand\ps{{(\Omega, \mathcal{F}, \mathbb{P})}}
\newcommand\R{\mathbb{R}}
\newcommand\Pb{\mathbb{P}}
\newcommand\s{$\sigma$\textrm{-field} }
\newcommand\notsubset{\subset\hspace{-3.5mm}{/}\hspace{1.5mm}}

\def\del{\dot{\Delta}}
\def\lap{\nabla}
\def\invf{\mathcal{F}^{-1}}
\def\dj{\delta_j}
\def\apo{(I + \delta B)^{-1}}
\def\T{\mathbb{T}}
\def\A{\mathrm{A}}
\def\mcal{\mathcal{M}}
\def\divv{\mathrm{div}}
\def\rE{\mathrm{E}}
\def\e{\mathbb{E}}
\newtheorem{theorem}{Theorem}[section]


%\newtheorem{theorem}{Theorem}[chapter]
\newtheorem{acknowledgement}[section]{Acknowledgement}
\newtheorem{alg}[theorem]{Algorithm}
\newtheorem{axiom}[theorem]{Axiom}
\newtheorem{case}[theorem]{Case}
\newtheorem{claim}[theorem]{Claim}
\newtheorem{conclusion}[theorem]{Conclusion}
\newtheorem{condition}[theorem]{Condition}
\newtheorem{conjecture}[theorem]{Conjecture}
\newtheorem{corollary}[theorem]{Corollary}
\newtheorem{criterion}[theorem]{Criterion}


\theoremstyle{definition}
\newtheorem{defn}[theorem]{Definition}
\theoremstyle{plain}

\newtheorem*{ass}{Assumptions}


\newtheorem{example1}[theorem]{Example}
\newtheorem{exercise1}[theorem]{Exercise}
\newtheorem{lemma}[theorem]{Lemma}
\newtheorem{notation}[theorem]{Notation}
\newtheorem{problem}[theorem]{Open Problem}
\newtheorem{proposition}[theorem]{Proposition}
\newtheorem{remark1}[theorem]{Remark}
\newtheorem{solution}[theorem]{Solution}
\newtheorem{summary}[theorem]{Summary}
\newtheorem{digression1}[theorem]{Digression}

%% The definitions below are modified from my SCBST lectures I replaced [Theorem] by [theorem]
\newtheorem{Definition}[theorem]{Definition}
\newtheorem{Remark}[theorem]{Remark}
\newtheorem{Proposition}[theorem]{Proposition}
\newtheorem{Lemma}[theorem]{Lemma}
\newtheorem{Corollary}[theorem]{Corollary}
\newtheorem{Exercise}[theorem]{Exercise}%[section]
\newtheorem{Example}[theorem]{Example}
\newtheorem{comments}{Comments}
\renewcommand{\thecomments}{}



%\newtheorem{rem}{Remark}
\renewcommand{\therem}{}
\newtheorem{note}{Notation}
\renewcommand{\thenote}{}

\newtheorem{example0}{Example}
\renewcommand{\theexample0}{}
%\newcommand\U{u^{\nu}}

\newtheorem{motivation}[theorem]{Motivation}



\newenvironment{definition}{
\begin{defn}
	\normalfont}{
\end{defn}
}

\newenvironment{algorithm}{
\begin{alg}
	\normalfont}{
\end{alg}
}

\newenvironment{remark}{
\begin{remark1}
	\normalfont}{
\end{remark1}
}

\newenvironment{example}{
\begin{example1}
	\normalfont}{
\end{example1}
}

\newenvironment{exercise}{
\begin{exercise1}
	\normalfont}{
\end{exercise1}
}


\newenvironment{digression}{
\begin{digression1}
	\normalfont}{
\end{digression1}
}

\newtheoremstyle{AppALem}{1}{1}
  {\itshape}{0pt}{\bfseries}{.}{ }
   {\thmname{Lemma }\thmnumber{A.{#2}}{\thmnote{}}}
   \theoremstyle{AppALem}\newtheorem{lemmaA}{Lemma}

\newtheoremstyle{AppBLem}{1}{1}
  {\itshape}{0pt}{\bfseries}{.}{ }
   {\thmname{Lemma }\thmnumber{B.{#2}}{\thmnote{}}}
   \theoremstyle{AppBLem}\newtheorem{lemmaB}{Lemma}
   
   \newtheoremstyle{AppCLem}{1}{1}
  {\itshape}{0pt}{\bfseries}{.}{ }
   {\thmname{Lemma }\thmnumber{C.{#2}}{\thmnote{}}}
   \theoremstyle{AppCLem}\newtheorem{lemmaC}{Lemma}
   
   \newtheoremstyle{AppDLem}{1}{1}
  {\itshape}{0pt}{\bfseries}{.}{ }
   {\thmname{Lemma }\thmnumber{D.{#2}}{\thmnote{}}}
   \theoremstyle{AppDLem}\newtheorem{lemmaD}{Lemma}


\newtheoremstyle{AppBRem}{1}{1}
  {\itshape}{0pt}{\bfseries}{.}{ }
   {\thmname{Remark }\thmnumber{B.{#2}}{\thmnote{}}}
   \theoremstyle{AppBRem}\newtheorem{remarkB}{Remark}

\newtheoremstyle{AppBCor}{1}{1}
  {\itshape}{0pt}{\bfseries}{.}{ }
   {\thmname{Corollary }\thmnumber{B.{#2}}{\thmnote{}}}
   \theoremstyle{AppBCor}\newtheorem{corB}[lemmaB]{Lemma}

\newtheoremstyle{AppBThm}{1}{1}
  {\itshape}{0pt}{\bfseries}{.}{ }
   {\thmname{Theorem }\thmnumber{B.{#2}}{\thmnote{}}}
   \theoremstyle{AppBThm}\newtheorem{theoremB}{Theorem}
   
\newtheoremstyle{AppCThm}{1}{1}
  {\itshape}{0pt}{\bfseries}{.}{ }
   {\thmname{Theorem }\thmnumber{C.{#2}}{\thmnote{}}}
   \theoremstyle{AppCThm}\newtheorem{theoremC}{Theorem}
   
   
   
\newcommand\coma[1]{{\color{red} #1}}
\newcommand\dela[2]{{\color{green}\sout{#1}#2}}
\newcommand\repa[2]{{\color{green}\sout{#1}}{\color{blue}{#2}}}
\newcommand\adda[1]{{\color{blue}#1}}
\newcommand\think[1]{}
\definecolor{darkred}{rgb}{0.9,0.1,0.1}%%%COLOR FOR SIDE COMMENT
\newcommand{\hcomment}[1]{\marginpar{\raggedright\scriptsize{\textcolor{darkred}{#1}}}}
\definecolor{darkblue}{rgb}{0.1,0.1,0.9}%%%COLOR FOR SIDE COMMENT
\newcommand{\hcommentg}[1]{\marginpar{\raggedright\scriptsize{\textcolor{darkblue}{#1}}}}
\frontmatter
%


\begin{titlingpage}
\begin{SingleSpace}
\calccentering{\unitlength} 
\begin{adjustwidth*}{\unitlength}{-\unitlength}
\vspace*{13mm}
\begin{center}
{\HUGE Preparation and Measurement Uncertainty in Quantum Mechanics}\\[4mm]
\vspace{31mm}
{\large By}\\
\vspace{9mm}
{\large\textsc{Oliver Owen Douglas Reardon-Smith}}\\
%\includegraphics[scale=0.2]{logos/UOY-Logo.png}\\
\vspace{31mm}
{\large
\textsc{Doctor of Philosophy}}\\
\vspace{21mm}
{\large
\textsc{University of York}\\
\textsc{Mathematics}
}\\
\vspace{21mm}
\vspace{9mm}
{\large\textsc{01/01/0001}}
\vspace{12mm}
\end{center}
\end{adjustwidth*}
\end{SingleSpace}
\end{titlingpage}
\clearemptydoublepage
%
%
\addtocontents{toc}{\par\nobreak \mbox{}\hfill{\textbf Page}\par\nobreak}
\pagenumbering{arabic}
\setcounter{page}{3}

%UoY:

%The abstract shall follow the title page. It shall provide a synopsis of the thesis, stating the nature and scope of work undertaken and the contribution made to knowledge in the subject treated. It shall appear on its own on a single page and shall not exceed 300 words in length. The abstract of the thesis may, after the award of the degree, be published by the University in any manner approved by the Senate, and for this purpose, the copyright of the abstract shall be deemed to be vested in the University.


\chapter*{Abstract}
\addcontentsline{toc}{chapter}{Abstract}
%\begin{SingleSpace}
This thesis addresses two forms of quantum uncertainty. The first, preparation uncertainty, is an expression of the fact that there are sets of observables for which the induced probability distributions are not simultaneously sharp in any state. We exactly characterise the preparation uncertainty regions for several finite dimensional case studies, including a new derivation of the preparation uncertainty region for the Pauli observables of qubits, and two qutrit case studies which have not previously been addressed in the literature.\\
We also consider the preparation uncertainty for observables relevant to multi-slit interferometry. We characterise the appropriate observables by their covariance properties. For one covariance condition the observables turn out to be equivalent to the position and momentum observables for a particle confined to move on a one dimensional ring, whilst for another they are equivalent to the canonical observables of a particle confined to move in a one dimensional ``box'', with infinite potential walls. A full characterisation of the uncertainty region for the particle on a ring is known. For the box we give bounds on the boundary of the region, and show that our upper bound is exact in an interval.\\
In the second part we turn our attention to measurement uncertainty, exploring the space of compatible joint approximations to incompatible target observables. We prove a quite general theorem, showing that for a broad class of figures of merit, the optimal compatible approximations to covariant targets are themselves covariant. This substantially simplifies the problem of determining measurement uncertainty regions for covariant observables, since the space of covariant compatible approximations is smaller than the space of all compatible approximations.\\
We employ this theorem to derive measurement uncertainty regions for three mutually orthogonal Pauli observables, and for the quantum Fourier pair acting in any finite dimension.
%\end{SingleSpace}
\clearpage
\clearemptydoublepage
%
\addcontentsline{toc}{chapter}{List of Contents}
\renewcommand{\contentsname}{List of Contents}
\maxtocdepth{subsection}
\tableofcontents*
\clearemptydoublepage
%
\listoftables
\addtocontents{lot}{\par\nobreak\textbf{{\scshape Table} \hfill Page}\par\nobreak}
\clearemptydoublepage
%
\listoffigures
\addtocontents{lof}{\par\nobreak\textbf{{\scshape Figure} \hfill Page}\par\nobreak}
\clearemptydoublepage
%

%

\chapter*{Acknowledgements and Dedication}
\addcontentsline{toc}{chapter}{Acknowledgements}
\pagenumbering{arabic}
\setcounter{page}{11}
%\begin{SingleSpace}

I would like to thank Paul Busch, who taught me so much, and helped me so much more. Roger Colbeck, for taking over my supervision when Paul no longer could and Stefan Weigert and Leon Loveridge for their patience, sympathy and support during a difficult time for all of us.\\
My research group, Peter Brown, Victoria Wright, Thomas Cope, Mirjam Weilenmann, Vilasini Venkatesh and Giorgos Eftaxias for many discussions, both useful and entertaining.\\
My friends, family, housemates and office-mates, and especially Peter Brown, Mirjam Weilenmann, Nicola Rendell, Allan Gerrard, Lucy Baker, Tom Quinn-Gregson and Rosie Leaman, who made some difficult times easier, and some easy times fantastic.\\
I am indebted to my collaborators, particuarly Jukka Kiukas whose clarity, rigour, patience and mathematical knowledge were critical to the work.\\
I acknowledge the generous support of EPSRC, and the University of York.\\ \\
This thesis is dedicated to the memory of Professor Paul Busch, who I am privileged to have called a mentor, and honoured to have called a friend. \\I will not be your last student.
%\end{SingleSpace}



\clearpage
\clearemptydoublepage
%
%
%
% UoB guidelines:
%
% Author's declaration
%
% I declare that the work in this dissertation was carried out in accordance
% with the requirements of the University's Regulations and Code of Practice
% for Research Degree Programmes and that it has not been submitted for any
% other academic award. Except where indicated by specific reference in the
% text, the work is the candidate's own work. Work done in collaboration with,
% or with the assistance of, others, is indicated as such. Any views expressed
% in the dissertation are those of the author.
%
% SIGNED: .............................................................
% DATE:..........................
%
\chapter*{Author's declaration}
\addcontentsline{toc}{chapter}{Author's Declaration}
\begin{SingleSpace}
\begin{quote}

\bigskip

\bigskip

\bigskip

\bigskip

\bigskip

\noindent{\large\textbf{List of publications and preprints}}\\
\end{quote}
\end{SingleSpace}
\clearpage
\clearemptydoublepage
\savepagenumber

%
% 
\mainmatter
\restorepagenumber
%
\import{chapter01/}{chap_intro.tex}
\clearemptydoublepage
\import{chapter02/}{chap_prelim.tex}
\clearemptydoublepage
\import{chapter_conclusion/}{conclusion.tex}
\clearemptydoublepage
%
%
%BIBLIOGRAPHY
\bibliographystyle{siam}

\begin{thebibliography}{100}


\end{thebibliography}

   
\end{document}