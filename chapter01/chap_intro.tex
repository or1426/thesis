\let\textcircled=\pgftextcircled
\chapter{Introduction and Synopsis}
\label{chap:intro}

In his seminal paper $92$ years ago Heisenberg brought attention to two different, but complementary, forms of uncertainty in quantum mechanics~\cite{Heisenberg1927-Wheeler+Zurek}. The first, which we will call \emph{preparation uncertainty}, refers to the fact that there exist sets of observables for which there are no states which make the probability distributions given by the Born rule simultaneously deterministic for every observable in the set. The standard description (see, for example~\cite{griffiths2005introduction})in textbooks of quantum mechanics of uncertainty follows this approach, generally focusing on the standard-deviation as a measure of the spread of a probability distribution, and exploring a tradeoff forcing one standard-deviation to become large as another becomes small. This idea has been explored and generalised extensively in the literature, for example by using different definitions of the spread of a probability measure~\cites{MaassenUffink1988}  \cites{doi:10.1063/1.3614503}{doi:10.1063/1.2759831}. We will explore generalisations of this idea using the concept of an \emph{uncertainty region} in the first part of this thesis.

In recent years it has become well known that the most famous thought experiment in Heisenberg's 1927 paper does not fit into this preparation focused view of uncertainty, and a new perspective has emerged which we will call \emph{measurement uncertainty}~\cites{PhysRevLett.111.160405}{blw-meas-uncertainty}{6773660Werner:2004:URJ:2011593.2011606}. A feature that separates quantum theory from classical is that quantum theory contains sets of observables for which there does not exist any joint observable, such observables are called \emph{incompatible} and the study of quantum incompatibility is a burgeoning field~\cites{Heinosaari_2016}{PhysRevLett.122.130402}{Heinosaari_2017}{PhysRevA.96.052127}. On the other hand it is possible, for example, to form approximations to the original observables by mixing them with trivial observables\footnote{Those for which the Born rule probability distribution is independent of the state.}, and at some level of mixing these approximations will become compatible~\cite{PhysRevA.87.052125}. We might broaden this view by considering arbitrary sets of compatible observables as approximators and, armed with some measure of the goodness of our approximations, investigate how closely we can approximate the original set.


\subsection*{Part~I -- Preparation uncertainty}



\subsection*{Part~2 -- Measurement uncertainty}

