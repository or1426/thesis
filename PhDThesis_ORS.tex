\RequirePackage[l2tabu]{nag}		% Warns for incorrect (obsolete) LaTeX usage
%
% Based on the following template, adapted by Gaurav Dhariwal and Mirjam Weilenmann to match the UoY guidelines.
% File: memoirthesis.tex
% Author: Victor Brena
% Description: Contains the thesis template using memoir class,
% which is mainly based on book class but permits better control of 
% chapter styles for example. This template is an adaptation and 
% modification of Oscar's.
% 
% Memoir is a flexible class for typesetting poetry, fiction, 
% non-fiction and mathematical works as books, reports, articles or
% manuscripts. CTAN repository is found at:
% http://www.ctan.org/tex-archive/macros/latex/contrib/memoir/
%
% Memoir class loads useful packages by default (see manual).
\documentclass[a4paper,11pt,leqno]{memoir} %add 'draft' to turn draft option on (see below)
%
%
% Adding metadata:
\usepackage{datetime}
\usepackage{ifpdf} 
\ifpdf
\pdfinfo{
   /Author Oliver Owen Douglas Reardon-Smith
   /Title (PhD Thesis)
   /Keywords (One; Two ; Three)
   /CreationDate (D:\pdfdate)
}
\fi
% When draft option is on. 
\ifdraftdoc 
	\usepackage{draftwatermark}				%Sets watermarks up.
	\SetWatermarkScale{0.3}
	\SetWatermarkText{\textbf Draft: \today}
\fi

\usepackage{subcaption}

% Better page layout for A4 paper, see memoir manual.
\settrimmedsize{297mm}{210mm}{*}
\setlength{\trimtop}{0pt} 
\setlength{\trimedge}{\stockwidth} 
\addtolength{\trimedge}{-\paperwidth} 
\settypeblocksize{634pt}{448.13pt}{*} 
\setulmargins{4cm}{*}{*} 
\setlrmargins{*}{*}{1.5} 
\setmarginnotes{17pt}{51pt}{\onelineskip} 
\setheadfoot{\onelineskip}{2\onelineskip} 
\setheaderspaces{*}{2\onelineskip}{*} 
\checkandfixthelayout
%

\usepackage{fouriernc}
\usepackage[T1]{fontenc}

\frenchspacing

%
\OnehalfSpacing 
%
% Sets numbering division level
\setsecnumdepth{subsection} 
\maxsecnumdepth{subsubsection}
%
% Chapter style (taken and slightly modified from Lars Madsen Memoir Chapter 
% Styles document
\usepackage{calc,soul,fourier}
\makeatletter 
\newlength\dlf@normtxtw 
\setlength\dlf@normtxtw{\textwidth} 
\newsavebox{\feline@chapter} 
\newcommand\feline@chapter@marker[1][4cm]{%
	\sbox\feline@chapter{% 
		\resizebox{!}{#1}{\fboxsep=1pt%
			\colorbox{black}{\color{white}\thechapter}% 
		}}%
		\rotatebox{90}{% 
			\resizebox{%
				\heightof{\usebox{\feline@chapter}}+\depthof{\usebox{\feline@chapter}}}% 
			{!}{\scshape\so\@chapapp}}\quad%
		\raisebox{\depthof{\usebox{\feline@chapter}}}{\usebox{\feline@chapter}}%
} 
\newcommand\feline@chm[1][4cm]{%
	\sbox\feline@chapter{\feline@chapter@marker[#1]}% 
	\makebox[0pt][c]{% aka \rlap
		\makebox[1cm][r]{\usebox\feline@chapter}%
	}}
\makechapterstyle{daleifmodif}{
	\renewcommand\chapnamefont{\normalfont\Large\scshape\raggedleft\so} 
	\renewcommand\chaptitlefont{\normalfont\Large\bfseries\scshape} 
	\renewcommand\chapternamenum{} \renewcommand\printchaptername{} 
	\renewcommand\printchapternum{\null\hfill\feline@chm[2.5cm]\par} 
	\renewcommand\afterchapternum{\par\vskip\midchapskip} 
	\renewcommand\printchaptertitle[1]{\color{black}\chaptitlefont\raggedright ##1\par}
} 
\makeatother 

\chapterstyle{daleifmodif}
%
\makepagestyle{myvf} 
\makeoddfoot{myvf}{}{\thepage}{}
\makeevenfoot{myvf}{}{\thepage}{} 
\makeheadrule{myvf}{\textwidth}{\normalrulethickness} 
\makeevenhead{myvf}{\tiny\textsc{\leftmark}}{}{} 
\makeoddhead{myvf}{}{}{\tiny\textsc{\rightmark}}
\pagestyle{myvf}
%
% Oscar's command (it works):
% Fills blank pages until next odd-numbered page. Used to emulate single-sided
% frontmatter. This will work for title, abstract and declaration. Though the
% contents sections will each start on an odd-numbered page they will
% spill over onto the even-numbered pages if extending beyond one page
% (hopefully, this is ok).
\newcommand{\clearemptydoublepage}{\newpage{\thispagestyle{empty}\cleardoublepage}}
%
%
% Creates indexes for Table of Contents, List of Figures, List of Tables and Index
\makeindex
% 
\usepackage{import}

\usepackage{color}

\usepackage[normalem]{ulem}


% Add other packages needed for chapters here. For example:
\usepackage{lipsum}					%Needed to create dummy text
\usepackage{amsfonts} 					%Calls Amer. Math. Soc. (AMS) fonts
\usepackage[centertags]{amsmath}			%Writes maths centred down
\usepackage{bm}
\usepackage{stmaryrd}					%New AMS symbols
\usepackage{amssymb}					%Calls AMS symbols
\usepackage{amsthm}					%Calls AMS theorem environment
\usepackage{newlfont}					%Helpful package for fonts and symbols
\usepackage{layouts}					%Layout diagrams
\usepackage{graphicx}					%Calls figure environment

\usepackage{longtable,rotating}			%Long tab environments including rotation. 
\usepackage[utf8]{inputenc}			%Needed to encode non-english characters 
					
%directly for mac
\usepackage{dsfont}									
\usepackage{colortbl}					%Makes coloured tables
\usepackage{wasysym}					%More math symbols
\usepackage{mathrsfs}					%Even more math symbols
\usepackage{esdiff}
\usepackage{mathtools}
\usepackage{float}						%Helps to place figures, tables, etc. 
\usepackage{verbatim}					%Permits pre-formated text insertion
\usepackage{upgreek}					%Calls other kind of greek alphabet
\usepackage{latexsym}					%Extra symbols

\usepackage[style=numeric-comp,maxbibnames=99]{biblatex}
\bibliography{bib_PhDThesis_ORS}

\usepackage{url}						%Supports url commands
\usepackage{etex}						%eTeXÕs extended support for counters

\usepackage[spanish,main=english,german,french]{babel}		%For languages characters and hyphenation
\usepackage{color}                    				%Creates coloured text and background
\usepackage[colorlinks=true,
		     allcolors=black]{hyperref}              %Creates hyperlinks in cross references
\usepackage{memhfixc}					%Must be used on memoir document 
									%class after hyperref
\usepackage{enumerate}					%For enumeration counter
\usepackage{footnote}					%For footnotes
\usepackage{microtype}					%Makes pdf look better.
\usepackage{rotfloat}					%For rotating and float environments as tables, 
									%figures, etc. 
\usepackage{alltt}						%LaTeX commands are not disabled in 
									%verbatim-like environment
\usepackage[version=0.96]{pgf}			%PGF/TikZ is a tandem of languages for producing vector graphics from a 
\usepackage{tikz}						%geometric/algebraic description.
\usetikzlibrary{arrows,shapes,snakes,
		       automata,backgrounds,
		       petri,topaths}				%To use diverse features from tikz		
%							
%Reduce widows  (the last line of a paragraph at the start of a page) and orphans 
% (the first line of paragraph at the end of a page)
\widowpenalty=1000
\clubpenalty=1000

\usepackage[bb=boondox]{mathalfa}

\usepackage{bbm} %to make bbm symbols
\usepackage[makeroom]{cancel}


\newcommand{\keywords}[1]{\par\noindent{\small{\textbf Keywords:} #1}} %Defines keywords small section
\newcommand{\parcial}[2]{\frac{\partial#1}{\partial#2}}                             %Defines a partial operator
\newcommand{\vectorr}[1]{\mathbf{#1}}                                                        %Defines a bold vector
\newcommand{\vecol}[2]{\left(                                                                         %Defines a column vector
	\begin{array}{c} 
		\displaystyle#1 \\
		\displaystyle#2
	\end{array}\right)}
\newcommand{\mados}[4]{\left(                                                                       %Defines a 2x2 matrix
	\begin{array}{cc}
		\displaystyle#1 &\displaystyle #2 \\
		\displaystyle#3 & \displaystyle#4
	\end{array}\right)}
\newcommand{\pgftextcircled}[1]{                                                                    %Defines encircled text
    \setbox0=\hbox{#1}%
    \dimen0\wd0%
    \divide\dimen0 by 2%
    \begin{tikzpicture}[baseline=(a.base)]%
        \useasboundingbox (-\the\dimen0,0pt) rectangle (\the\dimen0,1pt);
        \node[circle,draw,outer sep=0pt,inner sep=0.1ex] (a) {#1};
    \end{tikzpicture}
}

%for the tikz diagrams in low dim prep section
\usepackage{tikz}
\usepackage{tikz-3dplot}
\usetikzlibrary{positioning,intersections,shadings}
\usepackage{xstring}
\usepackage{pgfkeys}
\usepackage{xparse}

%Change tombstone symbol
\newcommand{\blackged}{\hfill$\blacksquare$}
\newcommand{\whiteged}{\hfill$\square$}
\newcounter{proofcount}
\renewenvironment{proof}[1][\proofname.]{\par
 \ifnum \theproofcount>0 \pushQED{\whiteged} \else \pushQED{\blackged} \fi%
 \refstepcounter{proofcount}
 \normalfont 
 \trivlist
 \item[\hskip\labelsep
       \itshape
   {\textbf\em #1}]\ignorespaces
}{%
 \addtocounter{proofcount}{-1}
 \popQED\endtrivlist
}

%My caption style
\newcommand{\mycaption}[2][\@empty]{
	\captionnamefont{\scshape} 
	\changecaptionwidth
	\captionwidth{0.9\linewidth}
	\captiondelim{.\:} 
	\indentcaption{0.75cm}
	\captionstyle[\centering]{}
	\setlength{\belowcaptionskip}{10pt}
	\ifx \@empty{#1} \caption{#2}\else \caption[{#1}]{{#2}}
}
%
% My subcaption style
\newcommand{\mysubcaption}[2][\@empty]{
	\subcaptionsize{\small}
	\hangsubcaption
	\subcaptionlabelfont{\rmfamily}
	\sidecapstyle{\raggedright}
	\setlength{\belowcaptionskip}{10pt}
	\ifx \@empty#1 \subcaption{#2}\else \subcaption[#1]{#2}
}

%
%An initial of the very first character of the content
\usepackage{lettrine}
\newcommand{\initial}[1]{%
	\lettrine[lines=3,lhang=0.33,nindent=0em]{
		\color{black}
     		{\textsc{#1}}}{}}
%
% Theorem styles used in my thesis
%
\theoremstyle{plain}
\newtheorem{thm}{Theorem}[chapter]
\theoremstyle{plain}
\newtheorem{prop}{Proposition}[chapter]
\theoremstyle{plain}
\theoremstyle{definition}
\newtheorem{dfn}{Definition}[chapter]
\theoremstyle{plain}
\newtheorem{lem}{Lemma}[chapter]
\theoremstyle{plain}
\newtheorem{cor}{Corollary}[chapter]
\theoremstyle{plain}
\newtheorem{rem}{Remark}[chapter]
%
% Hyphenation for some words
%
\hyphenation{res-pec-tively}
\hyphenation{mono-ti-ca-lly}
\hyphenation{hypo-the-sis}
\hyphenation{para-me-ters}
\hyphenation{sol-va-bi-li-ty}

\numberwithin{equation}{section}

%
%
\begin{document}
%

\renewcommand{\labelitemi}{$\bullet$}

\renewcommand{\qedsymbol}{$\blacksquare$}


\newcommand{\indep}{\rotatebox[origin=c]{90}{$\models$}}
\newcommand{\ot}{\otimes}
\newcommand{\eps}{\epsilon}
\newcommand{\sth}{ \ \mathrm{s.t.} \ }
\newcommand{\trans}{\mathrm{T}}
%\newcommand{\tr}[1]{\operatorname{tr}\left( #1 \right)}
\newcommand{\partr}[2]{\operatorname{tr}_{#2}\left( #1 \right)}
\newcommand{\trdist}[2]{\operatorname{D}\left( #1 | #2 \right)}
\newcommand{\lin}[1]{\mathcal{L}\left( #1 \right)} %linear operators...
\newcommand{\linH}[1]{\mathcal{L}_\mathrm{H}\left( #1 \right)}

\newcommand{\epi}{\operatorname{epi}}

%Quantum 
\newcommand{\ket}[1]{| #1 \rangle}
\newcommand{\bra}[1]{\langle #1 |}
\newcommand{\braket}[2]{\langle #1|#2\rangle}
\newcommand{\ketbra}[2]{|#1\rangle\!\langle#2|}
\newcommand{\proj}[1]{\ketbra{#1}{#1}}
\newcommand{\identity}{\openone}
\newcommand{\id}{\mathbbm{1}}
\newcommand{\cptp}{\mathcal{E}}

\newcommand{\acal}{\mathcal{A}}
\newcommand{\bcal}{\mathcal{B}}
\newcommand{\ccal}{\mathcal{C}}
\newcommand{\dcal}{\mathcal{D}}
\newcommand{\ecal}{\mathcal{E}}
\newcommand{\fcal}{\mathcal{F}}
\newcommand{\lcal}{\mathcal{L}}
\newcommand{\ocal}{\mathcal{O}}
\newcommand{\vcal}{\mathcal{V}}
\newcommand{\zcal}{\mathcal{Z}}
\newcommand{\kcal}{\mathcal{K}}
\newcommand{\hcal}{\mathcal{H}}
\newcommand{\tcal}{\mathcal{T}}
\newcommand{\scal}{\mathcal{S}}
\newcommand{\pcal}{\mathcal{P}}
\newcommand{\ucal}{\mathcal{U}}
%\newcommand{\fcal}{\mathcal{F}}

\newcommand{\sbuild}[2]{\left\{#1\,\middle|\,#2\right\}}

\newcommand{\supp}[1]{\operatorname{supp}\left({#1}\right)}

%operator spaces
\NewDocumentCommand{\bops}{ o o }
{
  \lcal\IfValueT{#1}{\!\left({#1}\IfValueT{#2}{,{#2}}\right)}
}

\NewDocumentCommand{\saops}{ o }
{
  \lcal_{s}\IfValueT{#1}{\!\left({#1}\right)}
}

\NewDocumentCommand{\posops}{ o }
{
  \lcal_{s}^{+}\IfValueT{#1}{\!\left({#1}\right)}
}

\NewDocumentCommand{\trops}{ o }
{
  \tcal\IfValueT{#1}{\!\left({#1}\right)}
}

\NewDocumentCommand{\dops}{ o }
{
  \scal\IfValueT{#1}{\!\left({#1}\right)}
}

\NewDocumentCommand{\probs}{ m m }
{
  \pcal\!\left({#1}, {#2}\right)
}

%sequence middle bit, variable, inf and sup
\NewDocumentCommand{\seq}{ m o o o }
{
  \left({#1}\right)_{ \IfValueT{#2}{#2\IfValueT{#3}{={#3}} }  }^{\IfValueT{#4}{#4}}
}

%\newcommand{\tr}[1]{\operatorname{tr}\left( #1 \right)}
%trace - optional partial trace
\NewDocumentCommand{\tr}{ m o }
{
  \operatorname{tr}{\IfValueT{#2}{_{#2}}}\left( #1 \right)
}

\newcommand{\vspan}[1]{\operatorname{span}\left(#1\right)}

\newcommand{\fmath}{\mathbb{F}}
\newcommand{\smath}{\mathbb{S}}
%\newcommand{\norm}[3]{\vert  #1 {\vert }_{#2}^{#3}} %%26/12/2013
\newcommand{\abs}[1]{{\left\lvert{#1}\right\rvert}}
\newcommand{\then}{\Rightarrow}
\newcommand\Eb{\mathbb{E}}
\newcommand\N{\mathbb{N}}
\newcommand\Z{\mathbb{Z}}
\newcommand\Q{\mathbb{Q}}
\newcommand\ps{{(\Omega, \mathcal{F}, \mathbb{P})}}
\newcommand{\R}{\mathbb{R}}
\newcommand{\K}{\mathbb{K}}
\newcommand\Pb{\mathbb{P}}
\newcommand{\exR}{\overline{\R}}
\newcommand{\Cx}{\mathbb{C}}
\newcommand{\T}{\mathbb{T}}
\newcommand\s{$\sigma$\textrm{-field} }
\newcommand\notsubset{\subset\hspace{-3.5mm}{/}\hspace{1.5mm}}

\def\del{\dot{\Delta}}
\def\lap{\nabla}
\def\invf{\mathcal{F}^{-1}}
\def\dj{\delta_j}
\def\apo{(I + \delta B)^{-1}}
\def\T{\mathbb{T}}
\def\A{\mathrm{A}}
\def\mcal{\mathcal{M}}
\def\divv{\mathrm{div}}
\def\rE{\mathrm{E}}
\def\e{\mathbb{E}}


%----------------------------------------------------------------------------------------------------------------------------------------------------------
%----------------------------------------------------------------------------------------------------------------------------------------------------------
%from low dimensional preparation ur paper

\newcommand{\born}{\bm{r}_0}

\newcommand{\boap}{\bm{\overline{a}}}
\newcommand{\vproj}[2]{\pi_{{#1}{#2}}}
\newcommand{\boa}{\bm{a}}
\newcommand{\bob}{\bm{b}}
\newcommand{\boc}{\bm{c}}
\newcommand{\bod}{\bm{d}}
\newcommand{\boe}{\bm{e}}
\newcommand{\bof}{\bm{f}}
\newcommand{\bog}{\bm{g}}
\newcommand{\boh}{\bm{h}}
%\newcommand{\boi}{\bm{i}} %overlap with Bable command 
\newcommand{\boj}{\bm{j}}
\newcommand{\bok}{\bm{k}}
\newcommand{\bol}{\bm{l}}
\newcommand{\bom}{\bm{m}}
\newcommand{\bon}{\bm{n}}
\newcommand{\boo}{\bm{o}}
\newcommand{\bop}{\bm{p}}
\newcommand{\boq}{\bm{q}}
\newcommand{\bor}{\bm{r}}
\newcommand{\bos}{\bm{s}}
%\newcommand{\bot}{\bm{t}} %overlap with the \perp symbol
\newcommand{\bou}{\bm{u}}
\newcommand{\bov}{\bm{v}}
\newcommand{\bow}{\bm{w}}
\newcommand{\boex}{\bm{x}}
\newcommand{\boy}{\bm{y}}
\newcommand{\boz}{\bm{z}}

\newcommand{\bogam}{\bm{\gamma}}
\newcommand{\boalpha}{\bm{\alpha}}


\newcommand{\bosig}{{\boldsymbol\sigma}}


\newcommand{\opa}{\operatorname{A}}
\newcommand{\opb}{\operatorname{B}}
\newcommand{\opc}{\operatorname{C}}
\newcommand{\opd}{\operatorname{D}}
\newcommand{\ope}{\operatorname{E}}
\newcommand{\opf}{\operatorname{F}}
\newcommand{\opg}{\operatorname{G}}
\newcommand{\oph}{\operatorname{H}}
\newcommand{\opi}{\operatorname{I}}
\newcommand{\opj}{\operatorname{J}}
\newcommand{\opk}{\operatorname{K}}
\newcommand{\opl}{\operatorname{L}}
\newcommand{\opm}{\operatorname{M}}
\newcommand{\opn}{\operatorname{N}}
\newcommand{\opo}{\operatorname{O}}
\newcommand{\opp}{\operatorname{P}}
\newcommand{\opq}{\operatorname{Q}}
\newcommand{\opr}{\operatorname{R}}
\newcommand{\ops}{\operatorname{S}}
\newcommand{\opt}{\operatorname{T}}
\newcommand{\opu}{\operatorname{U}}
\newcommand{\opv}{\operatorname{V}}
\newcommand{\opw}{\operatorname{W}}
\newcommand{\opx}{\operatorname{X}}
\newcommand{\opy}{\operatorname{Y}}
\newcommand{\opz}{\operatorname{Z}}
% \newcommand{\opf}{\mathrm{F}}
% \newcommand{\opj}{\mathrm{J}}
% \newcommand{\opx}{\mathrm{X}}
% \newcommand{\opy}{\mathrm{Y}}

\newcommand{\defeq}{:=}

\newcommand{\expr}[1]{\left\langle {#1}\right\rangle_\rho} % for the expectation value
\newcommand{\expnr}[1]{\left\langle {#1}\right\rangle} % for the expectation value without state argument



\DeclareDocumentCommand{\sdev}{ O{} m }{
  {\Delta_{{#1}} {#2}}
}
\DeclareDocumentCommand{\var}{ O{} m }{
  {\Delta^2_{#1} {#2}}
}


\DeclareDocumentCommand{\qbit}{ O{} O{+} O{\frac{1}{2}} m }{
  {{#3}\left({#1}\opi\, {#2}\, {#4}\cdot\bosig\right)}
}


\newcommand{\varr}[1]{{\var[\rho]{{#1}}}}
\newcommand{\sdevr}[1]{{\sdev[\rho]{{#1}}}}

\newcommand{\sdevmin}[1]{{\sdev{#1}_\text{min}}}
\newcommand{\varmin}[1]{{\var{#1}_\text{min}}}

\newcommand{\sdevmax}[1]{{\sdev{#1}_\text{max}}}
\newcommand{\varmax}[1]{{\var{#1}_\text{max}}}

\newcommand{\ft}[1]{\widetilde{#1}} % formal Fourier transform
\newcommand{\Cos}[1]{\cos\left(#1\right)} % cos with brackets
\newcommand{\Sin}[1]{\sin\left(#1\right)} % sin with brackets

\newcommand{\bracks}[1]{\left(#1\right)} % for brackets
\newcommand{\sbracks}[1]{\left[#1\right]} % square brackets
\newcommand{\com}[2]{\left[#1,#2\right]} % commutator
\newcommand{\comm}[2]{#1 #2 - #2 #1} % commutator spelled out
\newcommand{\acom}[2]{#1 #2 + #2 #1} % anti-commutator

\newcommand{\si}{\mathcal{S}}
\newcommand{\hi}{\mathcal{H}}

%\renewcommand{\max}{{\operatorname{max}}} % why do we need these??????
%\renewcommand{\min}{{\operatorname{min}}}

\renewcommand{\Re}{\operatorname{Re}}
\renewcommand{\Im}{\operatorname{Im}}
\newcommand{\ival}{I}
\newcommand{\PUR}[3]{\operatorname{PUR}_{#1}\left({#2},{#3}\right)}


\pgfkeys{
  /drawSemiCircle/.is family, /drawSemiCircle,
  defaults/.style = {scale = \textwidth,
    aAngle = 0,
    bAngle = 0,
    rAngle = 0,
    rLength = 1,
    drawBPrime = false,
    drawRightAngles = true,
    rightAngleScale = 0.07},
  scale/.estore in = \scale,
  aAngle/.estore in = \aAngle,
  bAngle/.estore in =\bAngle,
  rAngle/.estore in =\rAngle,
  rLength/.estore in =\rLength,
  drawBPrime/.estore in =\drawBPrime,
  rSub/.estore in =\rSub,
  drawRightAngles/.estore in =\drawRightAngles,
  rightAngleScale/.estore in =\rightAngleScale,
}



\newcommand{\DrawSemiCircle[2]}{%
  \pgfkeys{/drawSemiCircle, defaults, #1}%
  \begin{tikzpicture}[scale=\scale/2cm,>=stealth] %because a circle of radius 1 is 2cm wide this is the base length for the figure
    % draw the base line (before clipping so it doesn't get cut in half)
    \draw (-1,0) -- (1,0);

    % clip out the upper half
    \clip (-1cm, 0cm) rectangle (1.2cm, 1.2cm);

    % Draw the circle
    \draw (0cm,0cm) circle (1cm);

    % Put the a and b vectors on it
    \draw [name=vect,thick,->] (0,0) -- node[pos=1,fill=none,label=\aAngle:$\boa$] {} (\aAngle:1cm);
    \draw [name=vect,thick,->] (0,0) -- node[pos=1,fill=none,label=\bAngle:$\bob$] {} (\bAngle:1cm);
    \IfEqCase{\drawBPrime}{
      {true}{
        \pgfmathsetmacro{\bPrimeAngle}{\bAngle-90};
        \draw [name=vect,thick,->] (0,0) -- node[pos=1, left  = 0.1 ,fill=none] {$\bob^\prime$} (\bPrimeAngle:1cm);
        \IfEqCase{\drawRightAngles}{
          {true}{
            \pgfmathsetmacro{\midAngle}{(\bPrimeAngle+\bAngle)/2}
            \pgfmathsetmacro{\midLength}{sqrt(2)*\rightAngleScale}
            \draw [name=rangle] (\bAngle:\rightAngleScale) -- (\midAngle:\midLength) -- (\bPrimeAngle:\rightAngleScale);
          }
          {false}{}
        }
      }
      {false}{}
    }
    % now the r vector
    \draw [name=vect,thick,->] (0,0) -- node[midway, fill=white] {$\bor_{\rSub}$} (\rAngle:\rLength);

    % now we can compute r.a, r.b
    \pgfmathsetmacro{\ra}{1*\rLength*cos(\aAngle-\rAngle)}
    \pgfmathsetmacro{\rb}{1*\rLength*cos(\bAngle-\rAngle)}

    % and draw a*, b*, x and y
    \draw [name=vect,thick,->] (0,0) -\- node[pos=1, below right = 0.06 and 0.03 ,fill=none] {$\boa^*$} (\aAngle:\ra);
    \draw [name=vect,thick,->] (\aAngle:\ra) -- node[midway,fill=white] {$\bm{x}$} (\rAngle:\rLength);
    \IfEq{\rb}{0}{
      \IfEqCase{\drawRightAngles}{
        {true}{
          \pgfmathsetmacro{\midAngle}{(\rAngle+\bAngle)/2}
          \pgfmathsetmacro{\midLength}{sqrt(2)*\rightAngleScale}
          \draw [name=rangle] (\rAngle:\rightAngleScale) -- (\midAngle:\midLength) -- (\bAngle:\rightAngleScale);
        }
        {false}{}
      }
    }
    {\draw [name=vect,thick,->] (0,0) -- node[pos=1, below left  = 0.05 and 0.03 ,fill=none] {$\bob^*$} (\bAngle:\rb);
      \draw [name=vect,thick,->] (\bAngle:\rb) -- node[midway,fill=white] {$\bm{y}$} (\rAngle:\rLength);}

    % draw right angles between a*,x and b*,y
    \IfEq{\ra}{0}{}{
      \IfEqCase{\drawRightAngles}{
        {true}{
          \pgfmathsetmacro{\rx}{\rLength*cos(\rAngle)}
          \pgfmathsetmacro{\ry}{\rLength*sin(\rAngle)}

          \pgfmathsetmacro{\rax}{\ra*cos(\aAngle)}
          \pgfmathsetmacro{\ray}{\ra*sin(\aAngle)}
          \pgfmathsetmacro{\xxUnNormed}{\rx - \rax}
          \pgfmathsetmacro{\xyUnNormed}{\ry - \ray}
          \pgfmathsetmacro{\xx}{\xxUnNormed / sqrt(\xxUnNormed*\xxUnNormed + \xyUnNormed*\xyUnNormed)}
          \pgfmathsetmacro{\xy}{\xyUnNormed / sqrt(\xxUnNormed*\xxUnNormed + \xyUnNormed*\xyUnNormed)}
          \pgfmathsetmacro{\aStartX}{\rax * (1 + \rightAngleScale/\ra)}
          \pgfmathsetmacro{\aStartY}{\ray * (1 + \rightAngleScale/\ra)}
          \pgfmathsetmacro{\aMidX}{\aStartX  + \rightAngleScale*\xx}
          \pgfmathsetmacro{\aMidY}{\aStartY  + \rightAngleScale*\xy}
          \pgfmathsetmacro{\aEndX}{\rax + \rightAngleScale*\xx}
          \pgfmathsetmacro{\aEndY}{\ray + \rightAngleScale*\xy}
          \draw [name=rangle] (\aStartX,\aStartY) --  (\aMidX,\aMidY) -- (\aEndX,\aEndY);
        }
        {false}{}
      }
    }
    \IfEq{\rb}{0}{}{
      \IfEqCase{\drawRightAngles}{
        {true}{
          \pgfmathsetmacro{\rbx}{\rb*cos(\bAngle)}
          \pgfmathsetmacro{\rby}{\rb*sin(\bAngle)}
          \pgfmathsetmacro{\yxUnNormed}{\rx - \rbx}
          \pgfmathsetmacro{\yyUnNormed}{\ry - \rby}
          \pgfmathsetmacro{\yx}{\yxUnNormed / sqrt(\yxUnNormed*\yxUnNormed + \yyUnNormed*\yyUnNormed)}
          \pgfmathsetmacro{\yy}{\yyUnNormed / sqrt(\yxUnNormed*\yxUnNormed + \yyUnNormed*\yyUnNormed)}
          \pgfmathsetmacro{\bStartX}{\rbx * (1 + \rightAngleScale/\rb)}
          \pgfmathsetmacro{\bStartY}{\rby * (1 + \rightAngleScale/\rb)}
          \pgfmathsetmacro{\bMidX}{\bStartX  + \rightAngleScale*\yx}
          \pgfmathsetmacro{\bMidY}{\bStartY  + \rightAngleScale*\yy}
          \pgfmathsetmacro{\bEndX}{\rbx + \rightAngleScale*\yx}
          \pgfmathsetmacro{\bEndY}{\rby + \rightAngleScale*\yy}
          \draw [name=rangle] (\bStartX,\bStartY) --  (\bMidX,\bMidY) -- (\bEndX,\bEndY);
        }
        {false}{}
      }
    }
  \end{tikzpicture}
}
%correctly handles the cases where all the vectors are different, where b = r_0 and/or where b' = r_1 but no others (e.g. b' = r_1, b = a etc. will break the labels)
\DeclareDocumentCommand{\DrawSphere}{ m m m }{%
  \begin{tikzpicture}
    \tdplotsetmaincoords{10}{0}
    \tdplotsetrotatedcoords{90}{90}{-90}
    \coordinate (O) at (0,0,0);
    \pgfmathsetmacro{\scale}{0.1cm}

    \pgfmathsetmacro{\aAngle}{{#1}}
    \pgfmathsetmacro{\bAngle}{{#2}}
    \pgfmathsetmacro{\rZeroAngle}{{#3}}
    \pgfmathsetmacro{\rOneAngle}{\aAngle - abs(\aAngle - \rZeroAngle)}
    \pgfmathsetmacro{\rOneLength}{\rOneAngle < 0 ? sqrt( 1 + (cos(\aAngle - \bAngle) / cos(\bAngle - \aAngle + 90))^2  ) *  abs(cos(\rZeroAngle - \aAngle)) : 1}
    \pgfmathsetmacro{\rOneAngle}{\rOneAngle < 0 ? 0 : \rOneAngle}

    \pgfmathsetmacro{\bPrimeAngle}{\bAngle - 90}

    \tdplotsetcoord{a}{\scale}{90}{\aAngle}
    \tdplotsetcoord{b}{\scale}{90}{\bAngle}
    \tdplotsetcoord{bPrime}{\scale}{90}{\bPrimeAngle}
    \tdplotsetcoord{rZero}{\scale}{90}{\rZeroAngle}
    \tdplotsetcoord{rOne}{\rOneLength*\scale}{90}{\rOneAngle}

    \tdplotdrawarc[tdplot_screen_coords]{(O)}{\scale}{0}{360}{}{}

    %\draw [thick,->] (O) -- node[pos=1, right = 0.1, fill=none] {$\boa$} (a);
    \draw [thick,->] (O) -- node[pos=1,fill=none, label={\aAngle:$\boa$}] {} (a);
    \IfEq{\bAngle}{\rZeroAngle}{
      \draw [thick,->] (O) -- node[pos=1, fill=none, label={\bAngle:$\bob = \bor_0$}] {} (b);
    }{
      \draw [thick,->] (O) -- node[pos=1, fill=none, label={\bAngle:$\bob$}] {} (b);
      \draw [thick,->] (O) -- node[pos=1, fill=none, label={\rZeroAngle:$\bor_0$}] {} (rZero);
    }

    \IfEq{\bPrimeAngle}{\rOneAngle}{
      \IfEq{\rOneLength}{1}{
        \draw [thick,->] (O) -- node[pos=1, fill=none, label={\bPrimeAngle:$\bob^\prime = \bor_1$}] {} (bPrime);
      }{
        \draw [thick,->] (O) -- node[pos=1, fill=none, label={\bPrimeAngle:$\bob^\prime=\hat\bor_1$}] {} (bPrime);
        \draw [thick,->] (O) -- node[pos=1, fill=none, label={[label distance=0.2*\scale]120:$\bor_1$}] {} (rOne);
      }
    }{
      \draw [thick,->] (O) -- node[pos=1, fill=none, label={\bPrimeAngle:$\bob^\prime$}] {} (bPrime);
      \draw [thick,->] (O) -- node[pos=1, fill=none, label={\rOneAngle:$\bor_1$}] {} (rOne);
    }
    \pgfmathsetmacro{\bRZeroCircleRad}{abs(sin(\bAngle - \rZeroAngle))}
    \pgfmathsetmacro{\bRZeroCircleLoc}{abs(cos(\bAngle - \rZeroAngle))}
    \tdplotdrawarc[tdplot_rotated_coords]{(0,0,\bRZeroCircleLoc*\scale)}{\bRZeroCircleRad*\scale}{0}{180}{anchor=north}{  }
    \tdplotdrawarc[tdplot_rotated_coords, style=dashed]{(0,0,\bRZeroCircleLoc*\scale)}{\bRZeroCircleRad*\scale}{180}{360}{anchor=north}{  }

    \pgfmathsetmacro{\bROneCircleRad}{abs(sin(\bAngle - \rOneAngle))}
    \pgfmathsetmacro{\bROneCircleLoc}{cos(\bAngle - \rOneAngle)}
    \tdplotdrawarc[tdplot_rotated_coords]{(0,0,\bROneCircleLoc*\scale)}{\bROneCircleRad*\scale}{0}{180}{anchor=north}{  }
    \tdplotdrawarc[tdplot_rotated_coords, style=dashed]{(0,0,\bROneCircleLoc*\scale)}{\bROneCircleRad*\scale}{180}{360}{anchor=north}{  }

    \tdplotsetrotatedcoords{\aAngle}{90}{-90}
    \pgfmathsetmacro{\aRCircleRad}{abs(sin(\aAngle - \rZeroAngle))}
    \pgfmathsetmacro{\aRCircleLoc}{abs(cos(\aAngle - \rZeroAngle))}
    \tdplotdrawarc[tdplot_rotated_coords]{(0,0,\aRCircleLoc*\scale)}{\aRCircleRad*\scale}{0}{180}{anchor=north}{  }
    \tdplotdrawarc[tdplot_rotated_coords, style=dashed]{(0,0,\aRCircleLoc*\scale)}{\aRCircleRad*\scale}{180}{360}{anchor=north}{  }
  \end{tikzpicture}
}


%---------------------------------------------------------------------------------------------------------------------------------------------------------
% specific for cov-mur stuff
%----------------------------------------------------------------------------------------------------------------------------------------------------------
\newcommand{\grp}[3]{g\left({#1}, {#2}, {#3}\right)}
\newcommand{\map}[3]{R_{{#1}, {#2}, {#3}}}

\newcommand{\adj}[1]{{\operatorname{adj}\left({#1}\right)}}

\DeclareDocumentCommand{\cyc}{ O{n} }{
  {\mathbb{Z}_n}
}

\newcommand{\norm}[1]{{\left\lVert{#1}\right\rVert}}

\DeclareDocumentCommand{\pnorm}{ m O{p} }{
  \norm{{#1}}_{#2}
}
\newcommand{\fdiv}[3]{D_{#1}\left({#2}\parallel{#3}\right)}

\newcommand{\ovms}{\mathcal{M}}

%---------------------------------------------------------------------------------------------------------------------------------------------------------
% specific for interferometric prep stuff
%----------------------------------------------------------------------------------------------------------------------------------------------------------

\newcommand{\MOD}[1]{#1_{\mathrm{mod}}} % APP's Pmod and Qmod
\newcommand{\psdev}[1]{\tilde\Delta{\big(#1\big)}} % for the periodic standard deviation
%\newcommand{\ws}[1]{\delta{\big(#1\big)}} % for the delta-spread
\DeclareDocumentCommand{\ws}{ m o }{
  \operatorname{W}\IfValueT{ #2 }{_{#2}}{ } \left({#1}\right)
}

\DeclareDocumentCommand{\wsDev}{ m o }{
  \Delta\IfValueT{ #2 }{_{#2}}{ } \left({#1}\right)
}

\DeclareDocumentCommand{\lPVar}{ m m O{p} }{
  \delta_{#3}\left({#1}, {#2}\right)
}

\newcommand{\dom}[1]{\mathcal{D}(#1)}
\newcommand{\E}{\mathsf{E}} % spectral measure
\newcommand{\iunit}{\mathrm{i}} % for the imaginary unit i
%\newcommand{\diff}[2]{\frac{{\rm d}#1}{{\rm d }#2}} % diffs
\newcommand{\IT}{{I}_T} % for the interval [-T/2,T/2]
\newcommand{\IK}{{I}_K} % for the interval [-K/2,K/2]
\newcommand{\Ewp}{\mathsf E}
\newcommand{\Efc}{\mathsf F}

%pictures
\newcommand{\hh}{^H} %Heisenberg picture
\renewcommand{\ss}{^S} %Schroedinger picture

\newcommand{\expe}[2]{\left\langle{#1}\right\rangle_{#2}}

\DeclareDocumentCommand{\borel}{ m o }{
  \mathcal{B}\left({#1}\IfValueT{#2}{,{#2}} \right)
}

\newcommand{\width}[2]{W_{#2}\left({#1}\right)}

\DeclareDocumentCommand{\calibError}{ m m m o }{
  \Delta^{\IfValueTF{#4}{#4}{c}}_{#3}\left({#1}, {#2}\right)
}

\newcommand{\spec}[1]{\sigma\left({#1}\right)}
\newcommand{\res}[1]{\rho\left({#1}\right)}
\newtheorem{theorem}{Theorem}[section]


%\newtheorem{theorem}{Theorem}[chapter]
\newtheorem{acknowledgement}[section]{Acknowledgement}
\newtheorem{alg}[theorem]{Algorithm}
\newtheorem{axiom}[theorem]{Axiom}
\newtheorem{case}[theorem]{Case}
\newtheorem{claim}[theorem]{Claim}
\newtheorem{conclusion}[theorem]{Conclusion}
\newtheorem{condition}[theorem]{Condition}
\newtheorem{conjecture}[theorem]{Conjecture}
\newtheorem{corollary}[theorem]{Corollary}
\newtheorem{criterion}[theorem]{Criterion}
\newtheorem{defn}[theorem]{Definition}
\newtheorem*{ass}{Assumptions}

%\declaretheorem{axiom}
\newtheorem*{axiom_modified}{Axiom 7.3.4'}

\newtheorem{example1}[theorem]{Example}
\newtheorem{exercise1}[theorem]{Exercise}
\newtheorem{lemma}[theorem]{Lemma}
\newtheorem{notation}[theorem]{Notation}
\newtheorem{problem}[theorem]{Open Problem}
\newtheorem{proposition}[theorem]{Proposition}
\newtheorem{remark1}[theorem]{Remark}
\newtheorem{solution}[theorem]{Solution}
\newtheorem{summary}[theorem]{Summary}
\newtheorem{digression1}[theorem]{Digression}

%% The definitions below are modified from my SCBST lectures I replaced [Theorem] by [theorem]
\newtheorem{Definition}[theorem]{Definition}
\newtheorem{Remark}[theorem]{Remark}
\newtheorem{Proposition}[theorem]{Proposition}
\newtheorem{Lemma}[theorem]{Lemma}
\newtheorem{Corollary}[theorem]{Corollary}
\newtheorem{Exercise}[theorem]{Exercise}%[section]
\newtheorem{Example}[theorem]{Example}
\newtheorem{comments}{Comments}
\renewcommand{\thecomments}{}



%\newtheorem{rem}{Remark}
\renewcommand{\therem}{}
\newtheorem{note}{Notation}
\renewcommand{\thenote}{}

\newtheorem{example0}{Example}
\renewcommand{\theexample0}{}
%\newcommand\U{u^{\nu}}

\newtheorem{motivation}[theorem]{Motivation}



\newenvironment{definition}{
\begin{defn}
	\normalfont}{
\end{defn}
}

\newenvironment{algorithm}{
\begin{alg}
	\normalfont}{
\end{alg}
}

\newenvironment{remark}{
\begin{remark1}
	\normalfont}{
\end{remark1}
}

\newenvironment{example}{
\begin{example1}
	\normalfont}{
\end{example1}
}

\newenvironment{exercise}{
\begin{exercise1}
	\normalfont}{
\end{exercise1}
}


\newenvironment{digression}{
\begin{digression1}
	\normalfont}{
\end{digression1}
}

\newtheoremstyle{AppALem}{1}{1}
  {\itshape}{0pt}{\bfseries}{.}{ }
   {\thmname{Lemma }\thmnumber{A.{#2}}{\thmnote{}}}
   \theoremstyle{AppALem}\newtheorem{lemmaA}{Lemma}

\newtheoremstyle{AppBLem}{1}{1}
  {\itshape}{0pt}{\bfseries}{.}{ }
   {\thmname{Lemma }\thmnumber{B.{#2}}{\thmnote{}}}
   \theoremstyle{AppBLem}\newtheorem{lemmaB}{Lemma}
   
   \newtheoremstyle{AppCLem}{1}{1}
  {\itshape}{0pt}{\bfseries}{.}{ }
   {\thmname{Lemma }\thmnumber{C.{#2}}{\thmnote{}}}
   \theoremstyle{AppCLem}\newtheorem{lemmaC}{Lemma}
   
   \newtheoremstyle{AppDLem}{1}{1}
  {\itshape}{0pt}{\bfseries}{.}{ }
   {\thmname{Lemma }\thmnumber{D.{#2}}{\thmnote{}}}
   \theoremstyle{AppDLem}\newtheorem{lemmaD}{Lemma}


\newtheoremstyle{AppBRem}{1}{1}
  {\itshape}{0pt}{\bfseries}{.}{ }
   {\thmname{Remark }\thmnumber{B.{#2}}{\thmnote{}}}
   \theoremstyle{AppBRem}\newtheorem{remarkB}{Remark}

\newtheoremstyle{AppBCor}{1}{1}
  {\itshape}{0pt}{\bfseries}{.}{ }
   {\thmname{Corollary }\thmnumber{B.{#2}}{\thmnote{}}}
   \theoremstyle{AppBCor}\newtheorem{corB}[lemmaB]{Lemma}

\newtheoremstyle{AppBThm}{1}{1}
  {\itshape}{0pt}{\bfseries}{.}{ }
   {\thmname{Theorem }\thmnumber{B.{#2}}{\thmnote{}}}
   \theoremstyle{AppBThm}\newtheorem{theoremB}{Theorem}
   
\newtheoremstyle{AppCThm}{1}{1}
  {\itshape}{0pt}{\bfseries}{.}{ }
   {\thmname{Theorem }\thmnumber{C.{#2}}{\thmnote{}}}
   \theoremstyle{AppCThm}\newtheorem{theoremC}{Theorem}
   
   
   
\newcommand\coma[1]{{\color{red} #1}}
\newcommand\dela[2]{{\color{green}\sout{#1}#2}}
\newcommand\repa[2]{{\color{green}\sout{#1}}{\color{blue}{#2}}}
\newcommand\adda[1]{{\color{blue}#1}}
\newcommand\think[1]{}
\definecolor{darkred}{rgb}{0.9,0.1,0.1}%%%COLOR FOR SIDE COMMENT
\newcommand{\hcomment}[1]{\marginpar{\raggedright\scriptsize{\textcolor{darkred}{#1}}}}
\definecolor{darkblue}{rgb}{0.1,0.1,0.9}%%%COLOR FOR SIDE COMMENT
\newcommand{\hcommentg}[1]{\marginpar{\raggedright\scriptsize{\textcolor{darkblue}{#1}}}}
\frontmatter
%


\begin{titlingpage}
\begin{SingleSpace}
\calccentering{\unitlength} 
\begin{adjustwidth*}{\unitlength}{-\unitlength}
\vspace*{13mm}
\begin{center}
{\HUGE Preparation and Measurement Uncertainty in Quantum Mechanics}\\[4mm]
\vspace{31mm}
{\large By}\\
\vspace{9mm}
{\large\textsc{Oliver Owen Douglas Reardon-Smith}}\\
%\includegraphics[scale=0.2]{logos/UOY-Logo.png}\\
\vspace{31mm}
{\large
\textsc{Doctor of Philosophy}}\\
\vspace{21mm}
{\large
\textsc{University of York}\\
\textsc{Mathematics}
}\\
\vspace{21mm}
\vspace{9mm}
{\large\textsc{01/01/0001}}
\vspace{12mm}
\end{center}
\end{adjustwidth*}
\end{SingleSpace}
\end{titlingpage}
\clearemptydoublepage
%
%
\addtocontents{toc}{\par\nobreak \mbox{}\hfill{\textbf Page}\par\nobreak}
\pagenumbering{arabic}
\setcounter{page}{3}

%UoY:

%The abstract shall follow the title page. It shall provide a synopsis of the thesis, stating the nature and scope of work undertaken and the contribution made to knowledge in the subject treated. It shall appear on its own on a single page and shall not exceed 300 words in length. The abstract of the thesis may, after the award of the degree, be published by the University in any manner approved by the Senate, and for this purpose, the copyright of the abstract shall be deemed to be vested in the University.


\chapter*{Abstract}
\addcontentsline{toc}{chapter}{Abstract}
%\begin{SingleSpace}
This thesis addresses two forms of quantum uncertainty. In part \ref{part:prep-ur}, we focus on preparation uncertainty, an expression of the fact that there are sets of observables for which the induced probability distributions are not simultaneously sharp in any state. We exactly characterise the preparation uncertainty regions for several finite dimensional case studies, including a new derivation of the preparation uncertainty region for the Pauli observables of qubits, and two qutrit case studies which have not previously been addressed in the literature.\\
We also consider the variance based preparation uncertainty for position and momentum observables for the well known ``particle in  a box'' system. We see that the appropriate momentum observable is \emph{not} given by the spectral measure of a self-adjoint operator, although the position observable is. The box system lacks the phase-space symmetry used to determine the free particle and particle on a ring systems so determining the box uncertainty region is rather more difficult than in these cases. We give upper and lower bounds on the boundary of the uncertainty region, and show that our upper bound is exact in an interval.\\
In part \ref{part:meas-ur} we turn our attention to measurement uncertainty, exploring the space of compatible joint approximations to incompatible target observables. We prove a general theorem, which shows that, for a broad class of figures of merit, the optimal compatible approximations to covariant targets are themselves covariant. This substantially simplifies the problem of determining measurement uncertainty regions for covariant observables, since the space of covariant compatible approximations is smaller than the space of all compatible approximations.\\
We employ this theorem to derive measurement uncertainty regions for three mutually orthogonal Pauli observables, and for the quantum Fourier pair acting in any finite dimension.
%\end{SingleSpace}
\clearpage
\clearemptydoublepage
%
\addcontentsline{toc}{chapter}{List of Contents}
\renewcommand{\contentsname}{List of Contents}
\maxtocdepth{subsection}
\tableofcontents*
\clearemptydoublepage
%
\listoftables
\addtocontents{lot}{\par\nobreak\textbf{{\scshape Table} \hfill Page}\par\nobreak}
\clearemptydoublepage
%
\listoffigures
\addtocontents{lof}{\par\nobreak\textbf{{\scshape Figure} \hfill Page}\par\nobreak}
\clearemptydoublepage
%

%

\chapter*{Acknowledgements and Dedication}
\addcontentsline{toc}{chapter}{Acknowledgements}
\pagenumbering{arabic}
\setcounter{page}{11}
%\begin{SingleSpace}

I would like to thank Paul Busch, who taught me so much, and helped me so much more. Roger Colbeck, for taking over my supervision when Paul no longer could, and Stefan Weigert and Leon Loveridge for their patience, sympathy and support during a difficult time for all of us.\\
My research group, Peter Brown, Victoria Wright, Thomas Cope, Mirjam Weilenmann, Vilasini Venkatesh and Giorgos Eftaxias for many discussions, both useful and entertaining.\\
My family, for their endless support and peerless patience for $26$ years so far.\\
My friends, family, housemates and office-mates, and especially Peter Brown, Mirjam Weilenmann, Nicola Rendell, Allan Gerrard, Lucy Baker, Tom Quinn-Gregson, Luka Milic, Sam Whyman and, Rosie Leaman who made some difficult times easier, and some easy times fantastic.\\
I am indebted to my collaborators, particularly Jukka Kiukas whose clarity, rigour and mathematical knowledge were critical to the work.\\
I am deeply grateful to my examiners, Stefan Weigert and David Jennings, who made the process of finishing this work far more easy and enjoyable than I had any right for it to be.\\
I acknowledge the generous support of EPSRC, and the University of York.\\ \\
This thesis is dedicated to the memory of Professor Paul Busch, who I am privileged to have called a mentor, and honoured to have called a friend. \\I will not be your last student.
%\end{SingleSpace}



\clearpage 
\clearemptydoublepage
%
%
%
% UoB guidelines:
%
% Author's declaration
%
% I declare that the work in this dissertation was carried out in accordance
% with the requirements of the University's Regulations and Code of Practice
% for Research Degree Programmes and that it has not been submitted for any
% other academic award. Except where indicated by specific reference in the
% text, the work is the candidate's own work. Work done in collaboration with,
% or with the assistance of, others, is indicated as such. Any views expressed
% in the dissertation are those of the author.
%
% SIGNED: .............................................................
% DATE:..........................
%
\chapter*{Author's declaration}
\addcontentsline{toc}{chapter}{Author's Declaration}
\begin{SingleSpace}
\begin{quote}

I declare that the work presented in this thesis is original and my own, except where indicated by specific reference in the text, and that I am the sole author. The work was carried out under the supervision of Paul Busch and Roger Colbeck and has not been presented for any other academic award.

Chapter \ref{chap:prelim} contains background information, and is referenced as such. Chapter \ref{chap:low_dim_prep} is based on research done in collaboration with Paul Busch, and released as reference \cite{pb-ors-u-regions-u-relations}. Chapter \ref{chap:interferometric_prep_ur} is based on research done in collaboration with Jukka Kiukas. The material in chapter \ref{chap:cov-meas-ur} is my own, influenced by discussions with Paul Busch.

\bigskip

\bigskip

\bigskip

\bigskip

\bigskip

\noindent{\large\textbf{List of publications and preprints}}\\

\begin{itemize}
\item {\em Introduction to UniversalQCompiler}, \\ R. Iten, O. Reardon-Smith, L. Mondada, E. Redmond, R. S. Kohli, R. Colbeck, \\2019, preprint: \href{https://arxiv.org/abs/1904.01072}{[arXiv:quant-ph/1904.01072]}
\item {\em On Quantum Uncertainty Relations and Uncertainty Regions}, \\P. Busch, O. Reardon-Smith, \\2019, preprint: \href{https://arxiv.org/abs/1901.03695}{[arXiv:quant-ph/1901.03695]}
\end{itemize}


\end{quote}
\end{SingleSpace}
\clearpage
\clearemptydoublepage
\savepagenumber

%
% 
\mainmatter
\restorepagenumber
%
\import{chapter01/}{chap_intro.tex}
\clearemptydoublepage
\import{chapter02/}{chap_prelim.tex}
\clearemptydoublepage
\part{Preparation uncertainty}
\import{chapter03/}{chap_low_dim_prep_ur.tex}
\import{chapter04/}{chap_interferometric_ur.tex}
\clearemptydoublepage
\part{Measurement uncertainty}
\import{chapter05/}{chap_cov_mur.tex}
\clearemptydoublepage
\import{chapter_conclusion/}{conclusion.tex}
\clearemptydoublepage
%BIBLIOGRAPHY
\printbibliography

   
\end{document}