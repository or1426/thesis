

\let\textcircled=\pgftextcircled
\chapter{Introduction and Synopsis}
\label{chap:intro}

In his seminal paper $92$ years ago Heisenberg brought attention to two different, but complementary, forms of uncertainty in quantum mechanics~\cite{Heisenberg1927-Wheeler+Zurek}. The first, known as \emph{preparation uncertainty}, refers to the fact that there exist sets of observables for which there are no states which make the probability distributions given by the Born rule simultaneously deterministic for every observable in the set. The standard description of uncertainty (see, for example~\cite{griffiths2005introduction}) in textbooks of quantum mechanics follows this approach, generally focusing on the standard deviation as a measure of the spread of a probability distribution, and exploring a tradeoff forcing one standard deviation to become large as another becomes small. This idea has been explored and generalised extensively in the literature, for example by using different definitions of the spread of a probability measure~\cites{MaassenUffink1988}{doi:10.1063/1.3614503}{doi:10.1063/1.2759831}. We will explore further generalisations of this idea in the first part of this thesis.

In recent years it has become well known that the most famous thought experiment in Heisenberg's 1927 paper does not fit into this preparation focused view of uncertainty, and a new perspective has emerged known as \emph{measurement uncertainty}~\cites{PhysRevLett.111.160405}{blw-meas-uncertainty}{6773660Werner:2004:URJ:2011593.2011606}. A feature that separates quantum theory from classical is that quantum theory contains sets of observables for which there does not exist any joint observable. Such observables are called \emph{incompatible} and the study of quantum incompatibility is a burgeoning field~\cites{Heinosaari_2016}{PhysRevLett.122.130402}{Heinosaari_2017}{PhysRevA.96.052127}. On the other hand it is possible, for example, to form approximations to the original observables by mixing them with trivial observables\footnote{Those for which the Born rule probability distribution is independent of the state.}, and at some level of mixing these approximations will become compatible~\cite{PhysRevA.87.052125}. We might broaden this view by considering arbitrary sets of compatible observables as approximators and, armed with some measure of the goodness of our approximations, investigate how closely we can approximate the original set.


\section*{Chapter~\ref{chap:low_dim_prep}}

Here we investigate the concept of an uncertainty region, a slight generalisation of the usual concept of an uncertainty relation. We determine the variance-uncertainty region for all pairs, and an infinite family of triples of sharp $\pm 1$ valued observables. These case have been characterised in the literature~\cites{LiQiao2015}{AbbottAlzieuHallBranciard2016}, but we present new derivations, with a geometrical flavour. We also investigate the uncertainty regions of some qutrit systems, where the structure of states is more complex, and analytical bounds are more difficult to obtain. We determine the uncertainty region for a pair obtained by embedding two mutually unbiased qubit observables in a qutrit system, and that for a pair of Gell-Mann observables. We show that the uncertainty region for a pair of Pauli observables is entirely characterised by the Schr{\"o}dinger uncertainty relation. On the other hand we demonstrate that the Schr{\"o}dinger relation is insufficient to determine the uncertainty region for Gell-Mann observables.

\section*{Chapter~\ref{chap:infinite-prep-ur}}

Here we consider the preparation uncertainty for ``position'' and ``momentum'' observables associated with a standard example system, the so-called ``particle in a box'' system. The case of the free particle was addressed by Heisenberg and is well known, and uncertainty region for the particle on a ring was characterised by Busch, Kiukas and Werner~\cite{sharp-ur-num-angle}, and the free, but the box case does not seem to have been addressed in the literature. We show that it is necessary for the position representation wave-function to vanish at the boundary in order that the momentum variance is finite, although this assumption is often imposed due to the model that the walls of the box are infinite potential barriers. We obtain upper and lower bounds for the boundary curve of the uncertainty region of the particle in a box, and show that our upper bound is tight for an interval comprising slightly under two thirds\footnote{More precisely $1 - \sqrt{\frac{1}{3} - \frac{2}{\pi^2}} \approx 0.64$.} of the possible position variance values. 

\section*{Chapter~\ref{chap:cov-meas-ur}}

We formulate symmetries of quantum observables via a finite group, with action on the outcome set, and unital, linear representation acting on the effect space. We introduce a systematic approach for exploiting such symmetries via an ``invariant mean'' map, acting as a projection from the space of quantum observables to the subspace of covariant observables. We show that the ``invariant'' mean preserves compatibility of observables. Explicitly, a joint observable for the invariant means is given by the invariant mean of the Cartesian joint of the original observables. We apply this to show that for a wide range of figures of merit, the optimal compatible approximations to covariant observables are given by covariant observables. This simplifies the problem of characterising measurement uncertainty regions for compatible targets because the set of covariant compatible approximators is smaller than the set of all compatible approximators. We apply this framework to the measurement uncertainty region of three mutually unbiased Pauli observables, and that of the quantum Fourier pair in an arbitrary, finite dimensional space. Unknown to, and independently of, us this latter case was recently addressed by Werner~\cite{Werner2016} although his methods are rather different from ours, and specific to phase space observables.

We conjecture that the invariant mean map so defined may be generalised to the case of a compact Hausdorff topological group acting on a separable, locally compact metric space, where the average over group elements is replaced by an integral with respect to the Haar measure. Under the additional assumption that the observable that map is applied to is absolutely continuous with respect to a covariant observable.
