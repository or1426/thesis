\let\textcircled=\pgftextcircled
\chapter{Conclusions and Outlook} \label{chap:conclusion}


In part I of this thesis we examined preparation uncertainty in both finite and infinite dimensional contexts. We give a new, geometric, derivation of the uncertainty region for sharp $\pm 1$, valued observables. We also provide simple counterexamples demonstrating that the Schr\"odinger uncertainty relation does not suffice to fully characterise the uncertainty region for Hilbert spaces of dimension greater than $2$. We also compute the uncertainty region for a pair of qutrit observables arising as spectral measures of non-commuting operators, which commute on a subspace, and show that it contains the point where both probability measures are deterministic. \coma{probably contradicting something}

We also examine the preparation uncertainty for observables arising from an interferometric context. Here the solution depends explicitly on the topology of the outcome set chosen for the interference fringe observable. If we choose to assign it the topology, and group operation, of the circle group then we recover the case examined in \coma{cite}, where the full phase space symmetry allows an essentially complete description of the preparation uncertainty region. Conversely considering the interval embedded in $\R$, with the relative topology, we recover the case of a ``particle in a box'', well known to undergraduate physicists. In this case we lack the full phase space symmetry and consequently can not use the methods previously employed to analyse the uncertainty region. Nonetheless we obtain an upper bound on the boundary of the uncertainty and show it is exact in some interval. Explicitly it is exact for position uncertainties greater than some bound. We also show that the well known hyperbola, attributed to Heisenberg, is a lower bound for the boundary of the uncertainty region and that the difference between the upper and lower bound is small \coma{at most}, and converging to zero as the position uncertainty becomes small.

We do not have a proof that the only appropriate covariance conditions to impose in this case are the two considered in \coma{ref}, nor do we know of any alternatives. As we have seen, imposing different covariance conditions leads to very different uncertainty regions, so exploring the space of possible covariance conditions might be \coma{fruitful}. 

It would be interesting to consider the measurement uncertainty relation associated with these interferometric observables. Very little is known about measurement uncertainty in infinite dimensional systems in the absence of \coma{phase-space?} symmetry \coma{refs}. It would be interesting to see what could be achieved exploiting only the partial phase-space symmetry exhibited by the ``particle in a box'' setup.

Little is known about the general structure of preparation uncertainty regions, all examples known to the authors are star-convex, 

%=========================================================

In part II we provide a systematic framework for exploiting the covariance properties of a family of observables to determine their measurement uncertainty region. This framework may be applied arbitrary families of finite-dimensional observables and any distance measure on observables which is equivalent to taking the $\sup$ over quantum states of a metric on probability measures. \coma{talk about this in cov-mur section}

We apply this method to the case of a metric on observables based on the $p$-norm, although this is well suited to comparing observables with finite outcome sets there is no direct analogue for observables with infinitely many outcomes. Concretely, there is not a direct equivalent of the $p$-norm for probability measures, other than the case $p=1$, which recovers the total variation distance. The $f$-divergences, \coma{cite} are a family of ``distance'' measures on probability distributions, parameterised by a convex function $f$ such that $f(1) = 0$. With no additional constraints, the $f$-divergence $\delta_f$ is only defined for a pair of probability measures such that one is dominated by the other, in the sense that the support of one is entirely contained in the support of the other. If we attempt to follow the previous work and define an $f$-divergence for quantum observables by taking the $\sup$ over quantum states we will, in general, break this constraint. It might be interesting to characterise the class of functions $f$ for which this does not occur, a reasonable starting point would be those $f$ for which the $lim_{f\to\infty}f(x) = 0$.


